\chapter{ภาษาคอมพิวเตอร์}
%
\par{
การสื่อสารระหว่างคอมพิวเตอร์กับโปรแกรมเมอร์จะเป็นในลักษณะการสั่งงาน
แต่ในกรณีของโปรแกรมเมอร์ด้วยกันจะเป็นการแปลความหมายของกระบวนการการแก้ปัญหา
(Algorithm)
เพื่อศึกษาและตีความหมายการทำงานของกระบวนการแก้ปัญหาเพื่อตรวจสอบความถูกต้อง
เช่นเดียวกันกับภาษาศาสตร์
ภาษาคอมพิวเตอร์สามารถกำหนดความหมายของตัวภาษาได้โดยอาศัยองค์ประกอบ 
3 ส่วนคือ
\begin{enumerate}
%
\item \textbf{โครงสร้างของภาษา} (Syntax of language) 
เป็นส่วนที่ใช้ในการอธิบายเรื่องการรวมตัวของสัญลักษณ์ต่าง ๆ 
เพื่อประกอบกันเป็นคำหรือประโยคในภาษา 
โครงสร้างของภาษาจะกำหนดรูปแบบความสัมพันธ์ระหว่างส่วนต่าง ๆ 
ของภาษา เช่น ลำดับการเกิด 
ตัวอย่างเช่น ในภาษาคอมพิวเตอร์ทั่วไปเมื่อมีการใช้ \texttt{if} 
จะต้องตามด้วย \texttt{then} และอาจจะมี \texttt{else} 
หลังจาก \texttt{then} หรือไม่ก็ได้ เป็นต้น 
ฉะนั้นโครงสร้างของภาษาจะพิจารณาเฉพาะในเรื่องของโครงสร้าง 
หรือการรวมตัวกันของภาษาเท่านั้น
%
\item \textbf{ความหมายของภาษา}  (Semantic of language)  
เป็นส่วนที่อธิบายความหมายของโครงสร้างที่ถูกต้องของภาษา
ในภาษาคอมพิวเตอร์ความหมายของภาษา
หมายถึงการทำงานของคอมพิวเตอร์เมื่อทำงานตามคำสั่งของภาษาคอมพิวเตอร์
ซึ่งเราอาจจะอธิบายความหมายดังกล่าวในลักษณะของความสัมพันธ์กันระหว่างตัวแปรป้อนเข้ากับผลลัพธ์เมื่อคอมพิวเตอร์ทำงานแล้วเสร็จ
หรือลำดับขั้นตอนของการทำงานของคอมพิวเตอร์ที่ละขั้นตอน เป็นต้น
%
\item \textbf{ลักษณะใช้ของภาษา} (Pragmatic of language) 
จะเกี่ยวข้องกับผู้ใช้ภาษาในเรื่องของการใช้
ในภาษาคอมพิวเตอร์ส่วนในจะหมายถึงประเด็นเกี่ยวกับ
ประสิทธิภาพของโปรแกรมเมื่อพัฒนาโดยภาษาคอมพิวเตอร์
ความยากง่ายในการพัฒนา และรูปแบบของการพัฒนาโปรแกรม เป็นต้น
%
\end{enumerate}
%
เนื้อหาในบทนี้จะกล่าวถึงเฉพาะเรื่อง
โครงสร้างของภาษาและความหมายของภาษา
โดยเริ่มต้นที่โครงสร้างของภาษาก่อน 
เนื่องจากความหมายของภาษาจะถูกพิจารณาเมื่อโครงสร้างของภาษาถูกต้องตามหลักเกณฑ์ที่กำหนดไว้เสียก่อน
}
%
\section{โครงสร้างของภาษา}
%
\par{
โครงสร้างของภาษาคือลักษณะการรวมตัวของสัญลักษณ์ประกอบกันตามรูปแบบที่กำหนดไว้
โดยรูปแบบที่กำหนดไว้จะเรียกว่า ไวยากรณ์ของภาษา (Grammar)
ตัวอย่างเช่น ภาษา $A$ มีไวยากรณ์ของภาษา
กำหนดไว้ว่าจะต้องประกอบไปด้วยสัญลักษณ์ซึ่งเป็นสมาชิกเฉพาะในเซต 
$\{1,0\}$ เท่านั้นและลำดับของ ``$1$'' จะต้องมาก่อน ``$0$'' ทุกตัว
ดังนั้นถ้ากำหนดให้เซต $B = \{1, 10, 110, 111\}$ 
จะได้ว่าทุกสมาชิกใน $B$ เป็นภาษาที่เป็นซับเซตของภาษา $A$ 
เนื่องจากทุกสมาชิกในเซต $B$ เป็นไปตามไวยากรณ์ของภาษา $A$
}
%
\par{
จากตัวอย่างข้างต้นการอธิบายไวยากรณ์ของภาษาเป็นส่วนสำคัญของโครงสร้างของภาษาเนื่องจากเป็นการตรวจสอบถึงความถูกต้องของโครงสร้างภาษา
วิธีการรูปนัยสำหรับอธิบายไวยากรณ์จึงเป็นสิ่งจำเป็น
เนื่องจากวิธีการรูปนัยสามารถพิสูจน์ความถูกต้องได้โดยหลักการทางคณิตศาสตร์
การนิยามที่ \ref{defn_grammar} เป็นการนิยามไวยากรณ์ของภาษาแบบรูปนัย
%
\begin{defn}
\label{defn_grammar}
ไวยากรณ์ของภาษาจะประกอบไปด้วย 4 ส่วนลำดับ 
$\langle \Sigma, N, P, S\rangle$ โดยที่ 
$\Sigma$ (Terminal symbol) 
เป็นเซตจำกัดของสัญลักษณ์ที่สามารถปรากฎขึ้นได้ในภาษา 
$N$ (Nonterminal symbol) 
คือตัวแสดงลำดับของสัญลักษณ์ที่ยังไม่สมบูรณ์แต่ยังเป็นเพียงส่วนหนึ่งของสมาชิกในภาษาเท่านั้น 
$P$ แสดงกฎการสร้างสำหรับ 
$N$ เพื่อให้ได้มาซึ่งคำที่เป็นสมาชิกของภาษา และ 
$S$ เป็นสมาชิกหนึ่งใน $N$ เป็นตัวบอกจุดเริ่มต้นของการไวยากรณ์
\end{defn}
}
%
\par{
จากตัวอย่างภาษา $A$ ข้างต้น 
จะสามารถอธิบายส่วนของไวยากรณ์ของภาษา
โดยที่ $\Sigma$ คือ $\{0,1\}$ สำหรับ 
$N$ $P$ และ $S$ 
เราจะใช้รูปแบบ 
BNF (Backus Normal Form หรือ Backus–Naur Form)
ในการอธิบาย
}
%

\par{
BNF ถูกใช้ครั้งแรกในการอธิบายไวยากรณ์ของภาษา 
ALGOL 60 โดย 
John Backus และ Peter Naur 
โดยรูปแบบดังกล่าวจะใช้ 
``$\langle \rangle$'' 
สำหรับครอบตัวแสดงลำดับของสัญลักษณ์ที่ยังไม่สมบูรณ์
(เราจะเรียกทับศัพท์ว่าตัว Non-terminal)
ตัวอย่างเช่น
%
\begin{grammar}
<zero at end> ::= <all one>`0' 
\end{grammar}
%
``zero at end'' และ ``all one'' เป็นตัว Non-terminal
``0'' เป็นสัญลักษณ์ที่สามารถปรากฏได้ในภาษา
ส่วนสัญลักษณ์ ((เราจะเรียกทับศัพท์ว่าตัว Terminal) ``::='' 
เป็นส่วนหนึ่งของ BNF ใช้สำหรับอธิบายการเปลี่ยนแปลงของตัว Non-terminal
ซึ่งจะสามารถตีความได้ว่า ตัว Non-terminal ``zero at end'' คือ (หรือประกอบไปด้วย)
``all one'' ตามด้วย ตัว Terminal `0'
}
%
\par{
จะเห็นได้ว่า BNF ก็เป็นภาษาซึ่งมีหลักการในการตีความและโครงสร้างแบบง่าย ๆ 
ไม่ซับซ้อน มีหน้าที่ส่วนใหญ่ใช้ในการอธิบายไวยากรณ์ของภาษาอื่น ๆ 
เราจึงเรียก BNF ว่าเป็น \textit{อภิภาษา} (metalanguage)
}
%
\section{โครงสร้างนามธรรม (Abstract Syntax)}
%
\par{
การกำหนดความสัมพันธ์ขององค์ประกอบในภาษาคอมพิวเตอร์โดยทั่วไปจะสามารถกำหนดได้ในแบบที่ชัดเจน 
(Concrete Syntax) 
ซึ่งการกำหนดแบบชัดเจนจะบอกถึงกระบวนการตรวจสอบความถูกต้องของตัวภาษา 
ตัวอย่างเช่นตัวประโยคดำเนินการบวกและคูณของเลขตั้งแต่ 0 ถึง 9 
ซึ่งกำหนดให้เครื่องหมายคูณดึงตัวแปรไปใช้ก่อนเครื่องหมายบวก 
เราสามารถกำหนดโครงสร้างแบบชัดเจนได้ดังนี้
%
\begin{grammar}
<E> ::= <E> + <T> | <T> 

<T> ::= <T> $\times$ <NUM>  

<NUM> ::= 0 | 1 | 2 | 3 |$\ldots$|9 
\end{grammar}
%
}



