\chapter{การแสดงรูปแบบปัญหาเบื้องต้น}
%
\par{
ปัญหา (Problem) อาจสามารถแบ่งออกได้เป็น
\textit{ประเภทของปัญหา} (Problem Class)
กับ \textit{ตัวปัญหาที่กำลังพิจารณาอยู่} (Problem Instance)
ตัวอย่างเช่น กำหนดให้กราฟ $G(V,E)$ โดยที่ $V$ เป็นจุดยอดในกราฟ
และ $E \subseteq \{(v_i \in V,v_j \in V)\}$ เป็นเส้นเชื่อมระหว่างจุดยอดสองจุด 
ให้หาการใส่สีให้กับจุดยอดในกราฟ $G$ โดยที่จุดยอดที่มีเส้นเชื่อมถึงกันต้องมีสีต่างกัน 
ถือว่าเป็น ปัญหา โดยที่ 
\textit{ประเภทของปัญหา} คือ การใส่สีให้กับจุดยอดสองจุดที่มีเส้นเชื่อมระหว่างกันต้องเป็นคนละสี
และ \textit{ตัวปัญหาที่กำลังพิจารณาอยู่} คือ กราฟ $G(V,E)$
}
%
\par{
เมื่อพิจารณา ประเภทของปัญหา กับ ตัวปัญหาที่กำลังพิจารณาอยู่ 
จะสังเกตุได้ว่า ประเภทของปัญหา จะไม่เปลี่ยนแปลง ถึงแม้ว่าปัญหาจะเปลี่ยนไป 
ในขณะที่ตัวปัญหาที่กำลังพิจารณาจะเปลี่ยนแปลงขึ้นอยู่กับบริบทของปัญหา
ฉะนั้นถ้าเราสามารถแสดง 
ประเภทของปัญหาและตัวปัญหาที่กำลังพิจารณาอยู่ให้อยู่ในรูปแบบที่ประมวลผลได้
การแก้ปัญหาในแต่ละ ตัวปัญหา ก็จะง่ายขึ้น 
เนื่องจากเราแค่เปลี่ยนรูปแบบของตัวปัญหาเท่านั้น
จากตัวอย่างข้างต้น ถ้าเราสามารถแสดงรูปแบบของ ประเภทของปัญหา 
``การใส่สีให้กับจุดยอดสองจุดที่มีเส้นเชื่อมระหว่างกันต้องเป็นคนละสี''
การแก้ปัญหาของต้นจะเหลือเพียงการแสดงรูปแบบของ กราฟ $G(V,E)$ 
ซึ่งเป็นเรื่องที่ไม่ยากเท่าไหร่น่ะ
}
%
\par{
กำหนดให้ ปัญหา $P$ ประกอบด้วย ประเภทของปัญหา $C$ และ
ตัวปัญหาที่กำลังพิจารณาอยู่ $I$ จะเริ่มต้นด้วยการเปลี่ยน $C$ ให้กลายเป็น
กฎ (rule) และ $I$ เป็น ข้อเท็จจริง (fact)
เราจะเริ่มต้นด้วยกฏก่อน
}
%
\par{
จาก ประเภทของปัญหา $C$ การใส่สีให้กับจุดยอดสองจุดที่มีเส้นเชื่อมระหว่างกันต้องเป็นคนละสี
สามารถแสดงให้อยู่ในรูปแบบของ Answer Set Programming ได้ดังนี้
\begin{align*}
& 1 \{color(X,I) : c(I)\} 1 \leftarrow  v(X). \\
& \leftarrow color(X,I), color(Y,I), e(X,Y), c(I).
\end{align*}
จากกฏบรรทัดแรกเป็นการกำหนดค่าสี $color(X,I)$ 
ให้กับจุดยอดแต่ล่ะจุด $v(X)$ 
โดยขึ้นอยู่กับค่าสีที่กำหนดมาให้ $c(I)$
ส่วนในบรรทัดที่สองจะเป็นการตรวจสอบชุดของคำตอบที่ได้จากการสร้างจากกฎในข้อแรก
ซึ่งสามารถอธิบายตามความหมายได้ดังนี้
จุดยอด $X$ กับ $Y$ มีการกำหนดสีให้เป็น $I$
ไม่ได้ถ้า $X$ กับ $Y$ มีเส้นเชื่อมระหว่างกัน
}
%
\par{
เมื่อกำหนดรูปแบบของประเภทของปัญหาได้แล้ว 
ส่วนต่อมาก็จะเป็นการแสดงรูปแบบของ
ตัวปัญหาที่กำลังพิจารณาอยู่ 
ในกรณีของปัญหาการให้สี 
ปัญหาที่พิจารณาอยู่จะเป็นลักษณะหรือรูปร่างของกราฟและจำนวนสีที่สามารถกำหนดให้ได้
}
