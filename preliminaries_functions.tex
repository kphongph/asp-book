\chapter{ฟังก์ชั่นพื้นฐาน}
%
\par{
เนื้อหาในส่วนนี้จะเป็นใช้ $\lambda$ 
เพื่ออธิบายการความหมายของฟังก์ชั่นพื้นฐานที่จะใช้ต่อในบทต่อ ๆ ไป
ซึ่งจะประกอบไปด้วย ฟังก์ชันเอกลักษณ์ (Identity function)
}
%
\section{ฟังก์ชันเอกลักษณ์ (Identity function)}
%
\par{
\textit{ฟังก์ชันเอกลักษณ์} คือ 
ฟังก์ชันที่คืนค่าออกมาเป็นค่าเดิมจากตัวแปรที่ส่งค่าให้กลับฟังก์ชั่น
ดังแสดงความหมายตามนิยาม \ref{identify_function}
}
%
\begin{defn}
\label{identify_function}
กำหนดให้ $M$ เป็นเซตเซตหนึ่ง 
ฟังก์ชันเอกลักษณ์ $f$ บน $M$ 
คือ เซตของลำดับ $(m_i,m_j)$ 
โดยที่ $m_i,m_j \in M$ และ $m_i = m_j$ 
ซึ่งสามารถเขียนในรูปแบบของเซตได้ดังนี้
$f = \{(m_i \in M ,m_j \in M)| m_i = m_j\}$
\end{defn}
%
\par{
จากนิยาม \ref{identify_function} 
สามารถเขียนให้อยู่ในรูปแบบ $\lambda$-abstraction ได้ดังนี้
$\lambda x.x$ โดยที่มีตัวแปร $x$ ตัวแรกเป็น Bound identifier
และ $x$ ตัวที่สองเป็นส่วนของ $\lambda$-expression
จะเห็นได้ว่าเมื่อมีการส่งตัวแปรให้กับ $\lambda$-abstraction 
$\lambda x.x$ ตัวแปรดังกล่าวจะถูกพิจารณาว่าเป็น $x$ 
ซึ่งเป็น Bound identifier จากนั้นตัวแปรนั้นจะแทนที่ $x$ 
ซึ่งอยู่ในส่วนของ $\lambda$-expression ในที่นี่มีเพียง 
$x$ ตัวที่สองตัวเดียวเท่านั้น
ฉะนั้นผลลัพธ์ของการประเมินค่าของ $\lambda x.x$ 
จะได้ค่าของตัวแปรที่ให้กับฟังก์ชั่นเสมอ
}
%
\par{
ตัวอย่างการประเมินค่าของฟังก์ชั่น $\lambda x.x$ 
โดยที่กำหนดตัวแปรป้อนเข้าคือ 9 สามารถแสดงผลได้ดังนี้
%
\begin{align*}
(\lambda x.x) 9 & \Rightarrow 9
\end{align*}
%
ถ้าสมมุติให้ตัวแปรป้อนเข้าของ $\lambda x_1.x_1$ 
เป็นฟังก์ชั่นเอกลักษณ์เอง $\lambda x_2.x_2$ ผลลัพธ์ของการประเมินค่า
ก็จะได้ฟังก์ชั่นเอกสักษณ์ เนื่องจาก $x_1$ ซึ่งเป็น bound identifier ของ $\lambda$-abstraction $\lambda x_1.x_1$ 
จะถูกแทนที่ด้วยตัวแปรป้อนเข้า $\lambda x_2.x_2$
ดังแสดงลำดับการประเมินดังต่อไปนี้
%
\begin{align*}
(\lambda x_1.x_1)(\lambda x_2.x_2) & \Rightarrow (\lambda x_2.x_2) \\
& \Rightarrow (\lambda x.x) \\
\end{align*}
%
}

\section{ฟังก์ชั่นประกอบ (Composition function)}
\par{
\textit{ฟังก์ชั่นประกอบ} คือ ฟังก์ชั่นที่มาจากการรวมกันของฟังก์ชั่นสองฟังก์ชั่น 
โดยที่ตัวแปรป้อนกลับของฟังก์ชั่นหนึ่งเป็นตัวแปรป้อนเข้าของอีกฟังก์ชั่น 
ดังแสดงความหมายตามนิยาม \ref{composition_function}
%
\begin{defn}
\label{composition_function}
กำหนดให้ $X$ $Y$ และ $Z$ เป็นเซต 
ฟังก์ชั่น $f$ แสดงความสัมพันธ์จากเซต $X$ ไปยังเซต $Y$ 
ฟังก์ชั่น $g$ แสดงความสัมพันธ์จากเซต $ํY$ ไปยังเซต $Z$ 
ฉะนั้น $f \subseteq \{(x \in X, y \in Y)\}$
และ $g \subseteq \{(y \in Y, z \in Z)\}$
ฟังก์ชั่นประกอบ 
$g \circ f \subseteq \{(x \in X, z \in Z) | \exists (x,y \in Y) \in f ~\mbox{and}~ \exists(y \in Y,z) \in g \}$
\end{defn}
}

\par{
จากนิยาม \ref{composition_function}
สามารถเขียนให้อยู่ในรูปแบบ $\lambda$-abstraction ได้ดังนี้
$\lambda f.\lambda g. \lambda x.(f(g x))$
}

