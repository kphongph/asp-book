\documentclass[b5paper,12pt]{book}

%= Set Thai Language ==================================
%= ใช้ XeLaTeX

\usepackage[vcentering,dvips]{geometry}
\usepackage{subfigure}
\usepackage{pst-all}
\usepackage{xltxtra}
\usepackage{xunicode}
\usepackage{amsmath}
\usepackage[nounderscore]{syntax}
\usepackage{amsthm}
\usepackage{graphicx}
%\usepackage{xfrac}
\usepackage{algorithm}
\usepackage[noend]{algpseudocode}

%
%\usepackage{unicode-math}
\usepackage{fontspec}
\usepackage[font=footnotesize,labelfont=bf]{caption}
%\usepackage{mathspec}
%\renewcommand{lmr}{\fontfamily{lmr}\selectfont}
\usepackage{listings}

\lstset{
  basicstyle=\ttfamily,
  columns=fullflexible,
}

\usepackage{makeidx}

%\geometry{papersize={14.5cm,9in},total={4.8in,6.8in}}
%\usepackage[top=1.5cm,bottom=1.5cm,left=2cm,right=2cm]{geometry}

\XeTeXlinebreaklocale “th_TH” % สำหรับตัดคำ
\XeTeXlinebreakskip = 0pt plus 1pt %
\renewcommand{\baselinestretch}{1.2} % ตั้งระยะห่างระหว่างบรรทัด
\defaultfontfeatures{Scale=1.23}

\definecolor{lightgray}{rgb}{.9,.9,.9}
\definecolor{darkgray}{rgb}{.4,.4,.4}
\definecolor{purple}{rgb}{0.65, 0.12, 0.82}

\lstdefinelanguage{JavaScript}{
  keywords={typeof, new, true, false, catch, function, return, null, catch, switch, var, if, in, while, do, else, case, break},
  keywordstyle=\color{blue}\bfseries,
  ndkeywords={class, export, boolean, throw, implements, import, this},
  ndkeywordstyle=\color{darkgray}\bfseries,
  identifierstyle=\color{black},
  sensitive=false,
  comment=[l]{//},
  morecomment=[s]{/*}{*/},
  commentstyle=\color{purple}\ttfamily,
  stringstyle=\color{red}\ttfamily,
  morestring=[b]',
  morestring=[b]"
}


%\newfontfamily{\thaifont}[Script=Thai]{TH Sarabun New:script=thai}

\setromanfont[
BoldFont=THSarabun_Bold.ttf,
ItalicFont=THSarabun_Italic.ttf,
BoldItalicFont=THSarabun_Bold_Italic.ttf,
]{THSarabun.ttf}

%\setmathrm{Latin Modern Roman}

\newcommand{\lmr}{\fontfamily{lmr}\selectfont} % Latin Modern Roman
\newcommand{\lmss}{\fontfamily{lmss}\selectfont} % Latin Modern Sans
%\newcommand{\lmss}{\fontfamily{lmtt}\selectfont} % Latin Modern Mono
%\setmainfont[Script=Thai]{THSarabun.ttf}

\theoremstyle{definition}
\newtheorem{defn}{นิยาม}[chapter]
\newtheorem{conj}{Conjecture}[section]
\newtheorem{exmp}{Example}[section]

\renewcommand{\figurename}{รูปที่}
\renewcommand{\chaptername}{บทที่}

\newcounter{examplecounter}
\newenvironment{example}{%
  \bigskip\noindent
  \refstepcounter{examplecounter}%
  \textbf{ตัวอย่างปัญหา \theexamplecounter}% 
}{%
  \par\bigskip
}
\numberwithin{examplecounter}{chapter}

\title{Declarative Programing with Answer Set}
% Include Author name and Copyright holder name
\author{ผู้ช่วยศาสตราจารย์ ดร.พงศ์พันธ์ กิจสนาโยธิน}

\makeindex
% =====================================================

\begin{document}


%\frontmatter
%\maketitle
%\mainmatter

\tableofcontents

\chapter{บทนำ}
\par{
Answer Set Programming (ASP) \index{Answer Set Programming}
เป็นภาษาโปรแกรมเชิงประกาศ 
(Declarative Programming Language)
\index{Declarative Programming Language}
มีลักษณะการเขียนแตกต่างจะภาษาโปรแกรมเชิงกระบวนการ 
(Imperative Programming) 
\index{Imperative Programming Language}
ซึ่งจะเน้นเรื่องลำดับขั้นตอนการทำงานของเครื่องจักร 
ตัวอย่างของภาษาเชิงกระบวนการคือ ภาษาจาวา ภาษาซี เป็นต้น
ในขณะที่ภาษาโปรแกรมเชิงกระบวนการเน้นขั้นตอนการทำงาน 
การภาษาโปรแกรมเชิงประกาศนั้นจะเน้นที่จะอธิบายว่าปัญหาคืออะไร 
แล้วให้เครื่องจักรทำงานแก้ปัญหาดังกล่าวเองโดยไม่จำเป็นต้องบอกให้เครื่องจักรทำงานอย่างไรบ้าง
เพื่อเข้าใจในความแตกต่างของการพัฒนาภาษาทั้งสองรูปแบบ 
เราจะศึกษาตัวอย่างปัญหาเหยือกน้ำ
จากนั้นจะเริ่มอธิบายวิธีการแก้ปัญหาโดยวิธีการโปรแกรมแบบเชิงกระบวนการ
และตามด้วยการโปรแกรมเชิงประกาศ
}
%
\section{ปัญหาเหยือกน้ำ}
\label{sec_wjp}
%
\par{
ปัญหาเหยือกน้ำ (Water Jug Problem) 
\index{Water Jug Problem} กำหนดให้มีเหยือกน้ำ 2 ใบ
มีขนาดความจุ 4 ลิตร กับ 3 ลิตร ตามลำดับ
โดยที่เหยือกน้ำทั้งสองใบไม่มีขีดสำหรับวัดความจุของน้ำ
เป็นไปได้หรือไม่ว่าจะให้มีน้ำเหลืออยู่จำนวน 2 ลิตรในเหยือกขนาด 4 ลิตร
ถ้ากำหนดให้มีน้ำสำหรับใส่ในแต่ละเหยือกได้ตลอด
}
%
\par{
เราจะเริ่มด้วยแนวทางในการแก้ปัญหาเหยือกน้ำโดยไม่เกี่ยวข้องกับการโปรแกรมก่อน
ในการแก้ปัญหาจากตัวอย่างปัญหาเหยือกน้ำ
อันดับแรกคือการอธิบายสถานะของปัญหา
ซึ่งสถานะจะต้องรวมตัวแปรทุกตัวที่มีผลต่อผลลัพธ์ของปัญหา
จากปัญหาดังกล่าวสถานะของปัญหาสามารถอธิบายได้ด้วยจำนวนน้ำที่อยู่ในเหยือกน้ำแต่ละใบ
ฉะนั้นกำหนดให้ $w_3$ และ $w_4$ 
เป็นปริมาณน้ำที่อยู่ในเหยือกขนาด 3 และ 4 ลิตรตามลำดับ
ฉะนั้นสถานะของปัญหา ณ เวลา $t$ ใด ๆ 
สามารถอธิยายได้ด้วยปริมาณของน้ำในเหยือกทั้งสองใบในลักษณะเป็นลำดับดังนี้ 
$(w_3, w_4)_t$
}
%
\par{
เมื่อได้สถานะในการอธิบายลำดับของปัญหาแล้ว 
เราจะเริ่มกำหนดหรืออธิบายการเปลี่ยนสถานะของปัญหาจากสถานะหนึ่งไปอีกสถานะหนึ่ง
จากปัญหาดังกล่าว 
ถ้ากำหนดให้สถานะปัจจุบันคือ $(w_3, w_4)_t$ 
การเปลี่ยนสถานะหรือการเปลี่ยนแปลงปริมาณน้ำในแต่ละใบจะเกิดขึ้นได้สามแบบ
คือ (1) การเติมน้ำให้เต็มสำหรับเหยือกที่ยังมีน้ำไม่เต็ม
(2) การเทน้ำจากเหยือกใบหนึ่งไปยังอีกใบหนึ่ง และ (3) การเทน้ำในเหยือกทิ้ง
กรณีที่เป็นไปได้ทั้งหมดของ $(w_3',w_4')_{t+1}$ 
สามารถแสดงได้ดังต่อไปนี้
\begin{enumerate}
\item การเพิ่มน้ำจากแหล่งน้ำให้กับเหยือกน้ำใบใดใบหนึ่งจนเต็ม 
\begin{eqnarray}
(w_3',w_4') \leftarrow (3, w_4) \label{eq_wj3} \\
(w_3',w_4') \leftarrow (w_3, 4) \label{eq_wj4}
\end{eqnarray}
\item การถ่ายเทน้ำจากเหยือก 4 ลิตรไปยังเหยือก 3 ลิตร
\begin{itemize}
\item ถ้า $w_3 + w_4 \le 3$ 
\begin{eqnarray}
(w_3',w_4') \leftarrow (w_3+w_4, 0) \label{eq_wj4to3_1}
\end{eqnarray}
\item ถ้า $w_3 + w_4 > 3$ 
\begin{eqnarray}
(w_3',w_4') \leftarrow (3, w_4-(3-w_3)) \label{eq_wj4to3_2}
\end{eqnarray}
\end{itemize}
\item การถ่ายเทน้ำจากเหยือก 3 ลิตรไปยังเหยือก 4 ลิตร
\begin{itemize}
\item ถ้า $w_3 + w_4 \le 4$ 
\begin{eqnarray}
(w_3',w_4') \leftarrow (0, w_3+w_4) \label{eq_wj3to4_1}
\end{eqnarray}
\item ถ้า $w_3 + w_4 > 4$ 
\begin{eqnarray}
(w_3',w_4') \leftarrow (w_3-(4-w_4),4) \label{eq_wj3to4_2}
\end{eqnarray}
\end{itemize}
\item การเทน้ำทั้งหมดในเหยือกทิ้ง
\begin{eqnarray}
(w_3',w_4') \leftarrow (0, w_4) \label{eq_wje3} \\
(w_3',w_4') \leftarrow (w_3, 0) \label{eq_wje4}
\end{eqnarray}
\end{enumerate}
%
ถ้ากำหนดให้ $(w_3,w_4)_t = (0,0)_0$ จากการเปลี่ยนสถานะข้างต้น $(w_3',w_4')_{t+1}$ 
จะสามารถเป็นหนึ่งในสมาชิกของเซตต่อไปนี้ $\{(0,0)_1,(3,0)_1,(0,4)_1\}$ ซึ่ง 
$(0,0)_1$ เกิดขึ้นได้จากหนึ่งในกฏที่อธิบายด้วยสมการต่อไปนี้ สมการที่ (\ref{eq_wj3to4_1})  
สมการที่ (\ref{eq_wj4to3_1}) สมการที่ (\ref{eq_wje3}) และ 
สมการที่ (\ref{eq_wje4})
ส่วน $(3,0)_1$ และ $(0,4)_1$ เกิดขึ้นจาก 
สมการที่ (\ref{eq_wj3}) และ สมการที่ (\ref{eq_wj4}) ตามลำดับ
}
%
\par{
จะเห็นได้ว่าเมื่อพิจารณาเฉพาะการเปลี่ยนสถานะของปริมาณน้ำ สถานะที่ $(0,0)_0$ กับ สถานะที่ $(0,0)_1$
ไม่มีข้อแตกต่างกันเนื่องจากทั้งสองมีปริมาณน้ำในแต่ละเหยือกเท่ากัน  
ฉะนั้นเราจะไม่คิดเรื่องของเวลาสำหรับระบุลำดับของการเปลี่ยนแปลงสถานะของปัญหา 
(สาเหตุที่ผู้แต่งใช้ $t$ ก่อนหน้านี้เพื่อที่จะอธิบายลำดับของการเปลี่ยนแปลงสถานะของปัญหา)
จากการกำหนดรูปแบบสถานะ 
เราสามารถสร้างการกราฟแสดงสถานะทั้งหมดที่เป็นไปได้
ดังแสดงในรูปภาพที่ \ref{fg_allstatewj}
และตัวอย่างของการเปลี่ยนแปลงสถานะจากเริ่มต้นไปยังสถานะที่ต้องการได้ดังแสดงในรูปภาพที่ \ref{fg_transwj_1}
}
%
%
\begin{figure}[h!]
\centering
\begin{pspicture}(0,-0.5)(6,5)
\rput(0,4.5){\rnode{c00}{$(0,0)$}}
\rput(1.5,4.5){\rnode{c01}{$(0,1)$}}
\rput(3,4.5){\rnode{c02}{$(0,2)$}}
\rput(4.5,4.5){\rnode{c03}{$(0,3)$}}
\rput(6,4.5){\rnode{c04}{$(0,4)$}}
%
\rput(0,3){\rnode{c10}{$(1,0)$}}
\rput(1.5,3){\rnode{c11}{$(1,1)$}}
\rput(3,3){\rnode{c12}{$(1,2)$}}
\rput(4.5,3){\rnode{c13}{$(1,3)$}}
\rput(6,3){\rnode{c14}{$(1,4)$}}
%
\rput(0,1.5){\rnode{c20}{$(2,0)$}}
\rput(1.5,1.5){\rnode{c21}{$(2,1)$}}
\rput(3,1.5){\rnode{c22}{$(2,2)$}}
\rput(4.5,1.5){\rnode{c23}{$(2,3)$}}
\rput(6,1.5){\rnode{c24}{$(2,4)$}}
%
\rput(0,0){\rnode{c30}{$(3,0)$}}
\rput(1.5,0){\rnode{c31}{$(3,1)$}}
\rput(3,0){\rnode{c32}{$(3,2)$}}
\rput(4.5,0){\rnode{c33}{$(3,3)$}}
\rput(6,0){\rnode{c34}{$(3,4)$}}
%
\end{pspicture}
\caption{
แสดงสถานะที่เป็นไปได้ทั้งหมดของปริมาณน้ำในเหยือกแต่ละใบ 
ในรูปแบบ $(w_3,w_4)$ โดยที่ $w_3$ และ $w_4$ 
แสดงปริมาณน้ำเป็นลิตรในเหยือกขนาดความจุ 3 ลิตร และ ขนาดความจุ 
4 ลิตรตามลำดับ
}
\label{fg_allstatewj}
\end{figure}
%
\par{
จากรูปภาพที่ \ref{fg_transwj_1} 
เนื่องจากเราสามารถหาการเปลี่ยนแปลงสถานะจาก $(0,0)$ ไปยัง $(3,2)$ 
ได้
ซึ่งเป็นตอบปัญหาว่าสามารถมีน้ำจำนวน 2 ลิตรเหลือให้เหยือกขนาด 4 ลิตรได้โดยเริ่มต้นที่ไม่มีน้ำอยู่ในเหยือกทั้งสองใบเลย
}
%
\begin{figure}[t]
\centering
\begin{pspicture}(0,-0.5)(6,5)
\rput(0,4.5){\rnode{c00}{$(0,0)$}}
\rput(1.5,4.5){\rnode{c01}{$(0,1)$}}
\rput(3,4.5){\rnode{c02}{$(0,2)$}}
\rput(4.5,4.5){\rnode{c03}{$(0,3)$}}
\rput(6,4.5){\rnode{c04}{$(0,4)$}}
%
\rput(0,3){\rnode{c10}{$(1,0)$}}
%\rput(1.5,3){\rnode{c11}{$(1,1)$}}
%\rput(3,3){\rnode{c12}{$(1,2)$}}
%\rput(4.5,3){\rnode{c13}{$(1,3)$}}
\rput(6,3){\rnode{c14}{$(1,4)$}}
%
\rput(0,1.5){\rnode{c20}{$(2,0)$}}
%\rput(1.5,1.5){\rnode{c21}{$(2,1)$}}
%\rput(3,1.5){\rnode{c22}{$(2,2)$}}
%\rput(4.5,1.5){\rnode{c23}{$(2,3)$}}
\rput(6,1.5){\rnode{c24}{$(2,4)$}}
%
\rput(0,0){\rnode{c30}{$(3,0)$}}
\rput(1.5,0){\rnode{c31}{$(3,1)$}}
\rput(3,0){\rnode{c32}{$(3,2)$}}
\rput(4.5,0){\rnode{c33}{$(3,3)$}}
\rput(6,0){\rnode{c34}{$(3,4)$}}
%
\psset{nodesep=3pt,arrowsize=3pt 3}
\ncarc[arcangleB=30,arcangleA=30,linestyle=solid]{->}{c00}{c04}\mput*{1}
\ncline[linestyle=solid]{->}{c04}{c31}\mput*{2}
\ncline[linestyle=solid]{->}{c31}{c01}\mput*{3}
\ncline[linestyle=solid]{->}{c01}{c10}\mput*{4}
\ncline[linestyle=solid]{->}{c10}{c14}\mput*{5}
\ncline[linestyle=solid]{->}{c14}{c32}\mput*{6}
\end{pspicture}
\caption{แสดงการเปลี่ยนสถานะของปริมาณน้ำโดยเริ่มต้นจากที่มีน้ำ 0 
ลิตรบรรจุอยู่ในเหยือกทั้งสองใบ ไปยังคำตอบที่ต้องการคือมีน้ำเหลืออยู่ 2
ลิตรในเหยือกขนาด 4 ลิตร
โดยที่เราจะพิจารณาเฉพาะสถานะที่ไม่ซ้ำกันเท่านั้น 
ตัวอย่างเช่น จาก $(0,4)$ สามารถเปลี่ยนเป็น $(0,0)$ หรือ $(3,4)$ ได้
แต่เนื่องจากทั้งสองสถานะมีอยู่แล้วจึงไม่มีความจำเป็นต้องสร้างสถานะดังกล่าวใหม่}
\label{fg_transwj_1}
\end{figure}
%
%
%
\section{การโปรแกรมเชิงกระบวนการ}
%
%
%
\par{
ในเนื้อหาส่วนนี้เราจะใช้แนวคิดการโปรแกรมเชิงกระบวนการเพื่อพัฒนาชุดคำสั่งที่ใช้ในการแก้ปัญหาจากตัวอย่างปัญหาเหยือกน้ำซึ่งอธิยายก่อนหน้านี้ในหัวข้อที่ \ref{sec_wjp}
โดยที่การที่จะตอบปัญหาเหยือกน้ำได้เราจำเป็นต้องหาเส้นทางการเปลี่ยนสถานะจากจุดต้้งต้นไปยังสถานะที่ต้องการ 
ถ้าเส้นทางดังกล่าวสามารถเกิดขึ้นจากการเปลี่ยนสถานะที่กำหนดไว้คำตอบของปัญหาคือเป็นไปได้
ในทางกลับกันเราจำเป็นต้องการันตีว่าไม่มีเส้นทางดังกล่าวอยู่จริงถึงจะตอบได้ว่าเป็นไปไม่ได้
จะแนวคิดการแก้ปัญหาข้างต้นสามารถเปลี่ยนให้เป็น 
การค้นหาเส้นทางจากจุดหนึ่งไปยังอีกจุดหนึ่งได้ 
ซึ่งกระบวนการดังกล่าวสามารถอธิบายได้เป็นลำดับขั้นต่อไปนี้ 
%
\begin{enumerate}
\item \label{enum_wj_1} การกำหนดค่าเริ่มต้นของสถานะ $(w_3,w_4)$ ในที่นี่คือ $(0,0)$ 
\item ตรวจสอบว่า $(w_3,w_4)$ เป็นสถานะที่ต้องการหรือไม่ ในที่นี่คือ $(w_3,2)$ 
ถ้าใช่หยุดการทำงานแล้วตอบว่าทำได้
\item ทำการสร้างสถานะที่เป็นไปได้ทั้งหมด $N$ โดยที่ $N = generate(w_3,w_4)$
\item จากสถานะใน $N$ เพิ่มสถานะที่ยังไม่ได้ถูกพิจารณาแล้วนำไปเก็บไว้ที่ $W$
\item เปลี่ยนสถานะตัวแรกของ $W$ ให้เป็นค่าเริ่มต้นตัวใหม่แล้วเริ่มทำ (\ref{enum_wj_1}) 
ถ้าไม่มีสมาชิกเหลือใน $W$ คำตอบคือไม่สามารถทำได้
\end{enumerate}
%
จากลำดับขั้นตอนดังกล่าวข้างต้นเราสามารถพัฒนาโปรแกรมเชิงกระบวนการได้
ดังแสดงใน 
การแสดงขั้นตอนวิธีการ (Algorithm) \ref{al_wj}
กำหนดให้ $W$ เป็นเซตของสถานะที่รอการพิจารณา $C$ เป็นเซตของสถานะที่พิจารณาเรียบร้อยแล้ว
โดยที่ $N$ เป็นเซตของสถานะที่ถูกสร้างขึ้นจากการกำหนดสถานะที่กำลังพิจารณา 
ตัวอย่างเช่น ถ้ากำหนดให้สถานะที่กำลังพิจารณาเป็น $(1,1)$
ค่าของ $N$ จะมีค่าได้ดังนี้ $\{(3,1),(1,4),(0,1),(1,0),(2,0),(0,2)\}$
ซึ่งค่าของ $N$ จะได้มาจากกฎของการเปลี่ยนแปลงสถานะที่อธิบายตามสมการ \ref{eq_wj3}
จนถึง สมการ \ref{eq_wje4} โดยใช้ $(w_3,w_4)$ เป็นค่า $(1,1)$
}
%

%
%
\begin{algorithm}[t]
\lmr
\caption{Water Jug Problem from $(0,0)$ to $(w_3,2)$ \label{al_wj} }
\begin{algorithmic}[1]
\Function{WaterJug BFS}{}
\State $\textit{W} \gets \{(0,0)\}$
\State $\textit{C} \gets \{\}$
\Repeat
  \State $(w_3,w_4) \gets pop(W)$
  \If{$w_4 \neq 2$} 
    \State $\textit{append}((w_3,w_4),C)$
    \State $N \gets \textit{generate}(w_3,w_4)$
    \For{$(w_3',w_4') \in N$}
      \If{$(w_3',w_4') \notin (W \cup C)$}
        \State $\textit{append}((w_3',w_4'),W)$
      \EndIf
    \EndFor
  \EndIf
\Until{$w_4 = 2 \textit{~or~} W = \emptyset$}
\If{$w_4 = 2$} 
\State \Return true;
\EndIf
\State \Return false;
\EndFunction
\end{algorithmic}
\end{algorithm}
%
%
\par{
จะเห็นได้ว่าการพัฒนาโปรแกรมเชิงกระบวนการจะเน้นเรื่องลำดับขั้นตอนการทำงานอย่างชัดเจน
ซึ่งการกำหนดลำดับขั้นตอนก่อนหลังมีส่วนสำคัญในการหาคำตอบ 
ถ้ามีการเปลี่ยนลำดับของการพิจารณาการเปลี่ยนสถานะ 
(ลำดับของสมการ \ref{eq_wj3} กับ \ref{eq_wj4}) 
อาจนำมาซึ่งลำดับของการพิจารณาที่แตกต่างกันได้ ดังแสดงในรูปภาพที่ \ref{fg_transwj_2}
เป็นต้น
}
%
\begin{figure}[t]
\centering
\begin{pspicture}(0,-0.5)(6,5)
\rput(0,4.5){\rnode{c00}{$(0,0)$}}
\rput(1.5,4.5){\rnode{c01}{$(0,1)$}}
\rput(3,4.5){\rnode{c02}{$(0,2)$}}
\rput(4.5,4.5){\rnode{c03}{$(0,3)$}}
\rput(6,4.5){\rnode{c04}{$(0,4)$}}
%
\rput(0,3){\rnode{c10}{$(1,0)$}}
%\rput(1.5,3){\rnode{c11}{$(1,1)$}}
%\rput(3,3){\rnode{c12}{$(1,2)$}}
%\rput(4.5,3){\rnode{c13}{$(1,3)$}}
\rput(6,3){\rnode{c14}{$(1,4)$}}
%
\rput(0,1.5){\rnode{c20}{$(2,0)$}}
%\rput(1.5,1.5){\rnode{c21}{$(2,1)$}}
%\rput(3,1.5){\rnode{c22}{$(2,2)$}}
%\rput(4.5,1.5){\rnode{c23}{$(2,3)$}}
\rput(6,1.5){\rnode{c24}{$(2,4)$}}
%
\rput(0,0){\rnode{c30}{$(3,0)$}}
\rput(1.5,0){\rnode{c31}{$(3,1)$}}
\rput(3,0){\rnode{c32}{$(3,2)$}}
\rput(4.5,0){\rnode{c33}{$(3,3)$}}
\rput(6,0){\rnode{c34}{$(3,4)$}}
%
\psset{nodesep=3pt,arrowsize=3pt 3}
\ncarc[arcangleB=-35,arcangleA=-35,linestyle=solid]{->}{c00}{c30}\mput*{1}
\ncline[linestyle=solid]{->}{c30}{c03}\mput*{2}
\ncline[linestyle=solid]{->}{c03}{c33}\mput*{3}
\ncline[linestyle=solid]{->}{c33}{c24}\mput*{4}
\ncline[linestyle=solid]{->}{c24}{c20}\mput*{5}
\ncline[linestyle=solid]{->}{c20}{c02}\mput*{6}
\end{pspicture}
\caption{แสดงการเปลี่ยนสถานะของปริมาณน้ำโดยเริ่มต้นจากที่มีน้ำ 0 
ลิตรบรรจุอยู่ในเหยือกทั้งสองใบ 
โดยมีลำดับเส้นทางแตกต่างจาก รูปภาพที่ \ref{fg_transwj_1}}
\label{fg_transwj_2}
\end{figure}
%
\par{
การแสดงขั้นตอนวิธีการ (Algorithm) \ref{al_wj}
เราสามารถสร้างโปรแกรมการทำงานแบบเชิงกระบวนการด้วยภาษา Javascript บน NodeJS 
ที่ทำงานได้บนเครื่องคอมพิวเตอร์ทั่วไปที่มี NodeJS runtime อยู่ได้ดังแสดงต่อไปนี้
%
\begin{lstlisting}[
  xleftmargin=2em,framexleftmargin=1.5em,
  language=JavaScript,
  backgroundcolor=\color{lightgray},
  extendedchars=true,
  basicstyle=\footnotesize\ttfamily,
  showstringspaces=false,
  showspaces=false,
  numbers=left,
  numberstyle=\footnotesize\ttfamily,
  numbersep=5pt,
  tabsize=2,
  breaklines=true,
  showtabs=false,
  captionpos=b
]
var generate = function(ws) {
  var N = [];
  N.push({w3:3,w4:ws.w4});
  N.push({w3:ws.w3,w4:4});
  N.push({w3:0,w4:ws.w4});
  N.push({w3:ws.w3,w4:0});
  N.push({w3:ws.w3+ws.w4,w4:0});
  N.push({w3:0,w4:ws.w3+ws.w4});
  N.push({w3:3,w4:ws.w4-(3-ws.w3)});
  N.push({w3:ws.w3-(4-ws.w4),w4:4});
  return N;
}

var main = function(ws) {
  var C = [];
  var W = [ws];
  while(W.length!=0) {
    cws = W.pop();
    if(cws.w4==2) break;
    C.push(cws);
    N = generate(cws);
    for(var i=0;i<N.length;i++) {
      var e = N[i];
      if(e.w3>=0 && e.w3<=3 && e.w4>=0 && e.w4<=4) {
        var explored = false;
        for(var j=0;j<C.length;j++) {
          var k=C[j];
          if(k.w3==e.w3 && k.w4==e.w4) {
            explored = true;
            break;
          }
        }
        if(!explored) {
          W.push(e);
        }
      }
    }
  }
  if(cws.w4==2) {
    console.log('Yes');
  } else {
    console.log('No');
  }
}
\end{lstlisting}
%
การทำงานของโปรแกรมจะประกอบไปด้วย 2 ส่วนหลักคือ
การควบคุมการทำงานของการค้นหาสถานะที่ต้องการซึ่งอยู่ในส่วนของ \texttt{main}
กับ ส่วนการสร้างสถานะที่สามารถเป็นไปได้ทั้งหมดจากสถานะที่กำหนดให้ \texttt{generate}
ถึงแม้ว่าโปรแกรมดังกล่าวจะมีจำนวนบรรทัดที่สั้นสามารถอ่าน
และทำความเข้าใจได้โดยง่ายในเวลาไม่นานนักสำหรับนักโปรแกรมที่มีความชำนาญ
แต่เมื่อพิจารณาการเขียนโปรแกรมที่มีขนาดใหญ่หรือความซับซ้อนมากกว่านี้อาจจะต้องใช้เวลานาน
ในการทำความเข้าใจและสามารถพิจารณาความถูกต้องของการพัฒนาโปรแกรมได้ 
ในเรื่องของการพิจารณาความถูกต้องในที่นี่ 
หมายถึงการเขียนโปรแกรมให้เป็นไปตามขั้นตอนกระบวนการที่ระบุไว้
ผู้อ่านตัวโปรแกรมต้องพิจารณาในแต่ละขั้นตอนของโปรแกรมว่ามีการเปลี่ยนสถานะยังไง
มีผลยังไงกับผลลัพธ์ของการทำงาน
มีความสอดคล้องหรือขัดแย้งกับ การแสดงขั้นตอนวิธีการ (ดังแสดงใน Algorithm \ref{al_wj})
หรือไม่
ซึ่งเป็นเรื่องที่ทำได้แต่ไม่ง่าย เนื่องจากการแสดงขั้นตอนวิธีการไม่เหมือนกันกับการเขียนโปรแกรมจริง ๆ
ดังตัวอย่างที่ยกมาก่อนหน้านี้เป็นต้น
}
%
%
%
%
\section{การโปรแกรมเชิงประกาศ}
%
%
%


\begin{lstlisting}[xleftmargin=1em]
r(-10..10).
l(0..4).
w(0,0).
{w(0,Y),w(X,0),w(3,Y),w(X,4)} :- w(X,Y),l(X;Y).
{w(X+Y,0),w(0,X+Y)} :- w(X,Y),l(X;Y).
{w(X-(4-Y),4),w(3,Y-(3-X))} :- w(X,Y),l(X;Y).
goal :- w(X,2),l(X).
:-w(X,Y),r(X;Y),X>3.
:-w(X,Y),r(X;Y),X<0.
:-w(X,Y),r(X;Y),Y>4.
:-w(X,Y),r(X;Y),Y<0.
:-not goal.
\end{lstlisting}






\chapter{การแสดงรูปแบบปัญหาเบื้องต้น}
%
\par{
ปัญหา (Problem) อาจสามารถแบ่งออกได้เป็น
\textit{ประเภทของปัญหา} (Problem Class)
กับ \textit{ตัวปัญหาที่กำลังพิจารณาอยู่} (Problem Instance)
ตัวอย่างเช่น กำหนดให้กราฟ $G(V,E)$ โดยที่ $V$ เป็นจุดยอดในกราฟ
และ $E \subseteq \{(v_i \in V,v_j \in V)\}$ เป็นเส้นเชื่อมระหว่างจุดยอดสองจุด 
ให้หาการใส่สีให้กับจุดยอดในกราฟ $G$ โดยที่จุดยอดที่มีเส้นเชื่อมถึงกันต้องมีสีต่างกัน 
ถือว่าเป็น ปัญหา โดยที่ 
\textit{ประเภทของปัญหา} คือ การใส่สีให้กับจุดยอดสองจุดที่มีเส้นเชื่อมระหว่างกันต้องเป็นคนละสี
และ \textit{ตัวปัญหาที่กำลังพิจารณาอยู่} คือ กราฟ $G(V,E)$
}
%
\par{
เมื่อพิจารณา ประเภทของปัญหา กับ ตัวปัญหาที่กำลังพิจารณาอยู่ 
จะสังเกตุได้ว่า ประเภทของปัญหา จะไม่เปลี่ยนแปลง ถึงแม้ว่าปัญหาจะเปลี่ยนไป 
ในขณะที่ตัวปัญหาที่กำลังพิจารณาจะเปลี่ยนแปลงขึ้นอยู่กับบริบทของปัญหา
ฉะนั้นถ้าเราสามารถแสดง 
ประเภทของปัญหาและตัวปัญหาที่กำลังพิจารณาอยู่ให้อยู่ในรูปแบบที่ประมวลผลได้
การแก้ปัญหาในแต่ละ ตัวปัญหา ก็จะง่ายขึ้น 
เนื่องจากเราแค่เปลี่ยนรูปแบบของตัวปัญหาเท่านั้น
จากตัวอย่างข้างต้น ถ้าเราสามารถแสดงรูปแบบของ ประเภทของปัญหา 
``การใส่สีให้กับจุดยอดสองจุดที่มีเส้นเชื่อมระหว่างกันต้องเป็นคนละสี''
การแก้ปัญหาของต้นจะเหลือเพียงการแสดงรูปแบบของ กราฟ $G(V,E)$ 
ซึ่งเป็นเรื่องที่ไม่ยากเท่าไหร่น่ะ
}
%
\par{
กำหนดให้ ปัญหา $P$ ประกอบด้วย ประเภทของปัญหา $C$ และ
ตัวปัญหาที่กำลังพิจารณาอยู่ $I$ จะเริ่มต้นด้วยการเปลี่ยน $C$ ให้กลายเป็น
กฎ (rule) และ $I$ เป็น ข้อเท็จจริง (fact)
เราจะเริ่มต้นด้วยกฏก่อน
}
%
\par{
จาก ประเภทของปัญหา $C$ การใส่สีให้กับจุดยอดสองจุดที่มีเส้นเชื่อมระหว่างกันต้องเป็นคนละสี
สามารถแสดงให้อยู่ในรูปแบบของ Answer Set Programming ได้ดังนี้
\begin{align*}
& 1 \{color(X,I) : c(I)\} 1 \leftarrow  v(X). \\
& \leftarrow color(X,I), color(Y,I), e(X,Y), c(I).
\end{align*}
จากกฏบรรทัดแรกเป็นการกำหนดค่าสี $color(X,I)$ 
ให้กับจุดยอดแต่ล่ะจุด $v(X)$ 
โดยขึ้นอยู่กับค่าสีที่กำหนดมาให้ $c(I)$
ส่วนในบรรทัดที่สองจะเป็นการตรวจสอบชุดของคำตอบที่ได้จากการสร้างจากกฎในข้อแรก
ซึ่งสามารถอธิบายตามความหมายได้ดังนี้
จุดยอด $X$ กับ $Y$ มีการกำหนดสีให้เป็น $I$
ไม่ได้ถ้า $X$ กับ $Y$ มีเส้นเชื่อมระหว่างกัน
}
%
\par{
เมื่อกำหนดรูปแบบของประเภทของปัญหาได้แล้ว 
ส่วนต่อมาก็จะเป็นการแสดงรูปแบบของ
ตัวปัญหาที่กำลังพิจารณาอยู่ 
ในกรณีของปัญหาการให้สี 
ปัญหาที่พิจารณาอยู่จะเป็นลักษณะหรือรูปร่างของกราฟและจำนวนสีที่สามารถกำหนดให้ได้
}

%\chapter{ฟังก์ชั่นพื้นฐาน}
%
\par{
เนื้อหาในส่วนนี้จะเป็นใช้ $\lambda$ 
เพื่ออธิบายการความหมายของฟังก์ชั่นพื้นฐานที่จะใช้ต่อในบทต่อ ๆ ไป
ซึ่งจะประกอบไปด้วย ฟังก์ชันเอกลักษณ์ (Identity function)
}
%
\section{ฟังก์ชันเอกลักษณ์ (Identity function)}
%
\par{
\textit{ฟังก์ชันเอกลักษณ์} คือ 
ฟังก์ชันที่คืนค่าออกมาเป็นค่าเดิมจากตัวแปรที่ส่งค่าให้กลับฟังก์ชั่น
ดังแสดงความหมายตามนิยาม \ref{identify_function}
}
%
\begin{defn}
\label{identify_function}
กำหนดให้ $M$ เป็นเซตเซตหนึ่ง 
ฟังก์ชันเอกลักษณ์ $f$ บน $M$ 
คือ เซตของลำดับ $(m_i,m_j)$ 
โดยที่ $m_i,m_j \in M$ และ $m_i = m_j$ 
ซึ่งสามารถเขียนในรูปแบบของเซตได้ดังนี้
$f = \{(m_i \in M ,m_j \in M)| m_i = m_j\}$
\end{defn}
%
\par{
จากนิยาม \ref{identify_function} 
สามารถเขียนให้อยู่ในรูปแบบ $\lambda$-abstraction ได้ดังนี้
$\lambda x.x$ โดยที่มีตัวแปร $x$ ตัวแรกเป็น Bound identifier
และ $x$ ตัวที่สองเป็นส่วนของ $\lambda$-expression
จะเห็นได้ว่าเมื่อมีการส่งตัวแปรให้กับ $\lambda$-abstraction 
$\lambda x.x$ ตัวแปรดังกล่าวจะถูกพิจารณาว่าเป็น $x$ 
ซึ่งเป็น Bound identifier จากนั้นตัวแปรนั้นจะแทนที่ $x$ 
ซึ่งอยู่ในส่วนของ $\lambda$-expression ในที่นี่มีเพียง 
$x$ ตัวที่สองตัวเดียวเท่านั้น
ฉะนั้นผลลัพธ์ของการประเมินค่าของ $\lambda x.x$ 
จะได้ค่าของตัวแปรที่ให้กับฟังก์ชั่นเสมอ
}
%
\par{
ตัวอย่างการประเมินค่าของฟังก์ชั่น $\lambda x.x$ 
โดยที่กำหนดตัวแปรป้อนเข้าคือ 9 สามารถแสดงผลได้ดังนี้
%
\begin{align*}
(\lambda x.x) 9 & \Rightarrow 9
\end{align*}
%
ถ้าสมมุติให้ตัวแปรป้อนเข้าของ $\lambda x_1.x_1$ 
เป็นฟังก์ชั่นเอกลักษณ์เอง $\lambda x_2.x_2$ ผลลัพธ์ของการประเมินค่า
ก็จะได้ฟังก์ชั่นเอกสักษณ์ เนื่องจาก $x_1$ ซึ่งเป็น bound identifier ของ $\lambda$-abstraction $\lambda x_1.x_1$ 
จะถูกแทนที่ด้วยตัวแปรป้อนเข้า $\lambda x_2.x_2$
ดังแสดงลำดับการประเมินดังต่อไปนี้
%
\begin{align*}
(\lambda x_1.x_1)(\lambda x_2.x_2) & \Rightarrow (\lambda x_2.x_2) \\
& \Rightarrow (\lambda x.x) \\
\end{align*}
%
}

\section{ฟังก์ชั่นประกอบ (Composition function)}
\par{
\textit{ฟังก์ชั่นประกอบ} คือ ฟังก์ชั่นที่มาจากการรวมกันของฟังก์ชั่นสองฟังก์ชั่น 
โดยที่ตัวแปรป้อนกลับของฟังก์ชั่นหนึ่งเป็นตัวแปรป้อนเข้าของอีกฟังก์ชั่น 
ดังแสดงความหมายตามนิยาม \ref{composition_function}
%
\begin{defn}
\label{composition_function}
กำหนดให้ $X$ $Y$ และ $Z$ เป็นเซต 
ฟังก์ชั่น $f$ แสดงความสัมพันธ์จากเซต $X$ ไปยังเซต $Y$ 
ฟังก์ชั่น $g$ แสดงความสัมพันธ์จากเซต $ํY$ ไปยังเซต $Z$ 
ฉะนั้น $f \subseteq \{(x \in X, y \in Y)\}$
และ $g \subseteq \{(y \in Y, z \in Z)\}$
ฟังก์ชั่นประกอบ 
$g \circ f \subseteq \{(x \in X, z \in Z) | \exists (x,y \in Y) \in f ~\mbox{and}~ \exists(y \in Y,z) \in g \}$
\end{defn}
}

\par{
จากนิยาม \ref{composition_function}
สามารถเขียนให้อยู่ในรูปแบบ $\lambda$-abstraction ได้ดังนี้
$\lambda f.\lambda g. \lambda x.(f(g x))$
}


%\chapter{ภาษาคอมพิวเตอร์}
%
\par{
การสื่อสารระหว่างคอมพิวเตอร์กับโปรแกรมเมอร์จะเป็นในลักษณะการสั่งงาน
แต่ในกรณีของโปรแกรมเมอร์ด้วยกันจะเป็นการแปลความหมายของกระบวนการการแก้ปัญหา
(Algorithm)
เพื่อศึกษาและตีความหมายการทำงานของกระบวนการแก้ปัญหาเพื่อตรวจสอบความถูกต้อง
เช่นเดียวกันกับภาษาศาสตร์
ภาษาคอมพิวเตอร์สามารถกำหนดความหมายของตัวภาษาได้โดยอาศัยองค์ประกอบ 
3 ส่วนคือ
\begin{enumerate}
%
\item \textbf{โครงสร้างของภาษา} (Syntax of language) 
เป็นส่วนที่ใช้ในการอธิบายเรื่องการรวมตัวของสัญลักษณ์ต่าง ๆ 
เพื่อประกอบกันเป็นคำหรือประโยคในภาษา 
โครงสร้างของภาษาจะกำหนดรูปแบบความสัมพันธ์ระหว่างส่วนต่าง ๆ 
ของภาษา เช่น ลำดับการเกิด 
ตัวอย่างเช่น ในภาษาคอมพิวเตอร์ทั่วไปเมื่อมีการใช้ \texttt{if} 
จะต้องตามด้วย \texttt{then} และอาจจะมี \texttt{else} 
หลังจาก \texttt{then} หรือไม่ก็ได้ เป็นต้น 
ฉะนั้นโครงสร้างของภาษาจะพิจารณาเฉพาะในเรื่องของโครงสร้าง 
หรือการรวมตัวกันของภาษาเท่านั้น
%
\item \textbf{ความหมายของภาษา}  (Semantic of language)  
เป็นส่วนที่อธิบายความหมายของโครงสร้างที่ถูกต้องของภาษา
ในภาษาคอมพิวเตอร์ความหมายของภาษา
หมายถึงการทำงานของคอมพิวเตอร์เมื่อทำงานตามคำสั่งของภาษาคอมพิวเตอร์
ซึ่งเราอาจจะอธิบายความหมายดังกล่าวในลักษณะของความสัมพันธ์กันระหว่างตัวแปรป้อนเข้ากับผลลัพธ์เมื่อคอมพิวเตอร์ทำงานแล้วเสร็จ
หรือลำดับขั้นตอนของการทำงานของคอมพิวเตอร์ที่ละขั้นตอน เป็นต้น
%
\item \textbf{ลักษณะใช้ของภาษา} (Pragmatic of language) 
จะเกี่ยวข้องกับผู้ใช้ภาษาในเรื่องของการใช้
ในภาษาคอมพิวเตอร์ส่วนในจะหมายถึงประเด็นเกี่ยวกับ
ประสิทธิภาพของโปรแกรมเมื่อพัฒนาโดยภาษาคอมพิวเตอร์
ความยากง่ายในการพัฒนา และรูปแบบของการพัฒนาโปรแกรม เป็นต้น
%
\end{enumerate}
%
เนื้อหาในบทนี้จะกล่าวถึงเฉพาะเรื่อง
โครงสร้างของภาษาและความหมายของภาษา
โดยเริ่มต้นที่โครงสร้างของภาษาก่อน 
เนื่องจากความหมายของภาษาจะถูกพิจารณาเมื่อโครงสร้างของภาษาถูกต้องตามหลักเกณฑ์ที่กำหนดไว้เสียก่อน
}
%
\section{โครงสร้างของภาษา}
%
\par{
โครงสร้างของภาษาคือลักษณะการรวมตัวของสัญลักษณ์ประกอบกันตามรูปแบบที่กำหนดไว้
โดยรูปแบบที่กำหนดไว้จะเรียกว่า ไวยากรณ์ของภาษา (Grammar)
ตัวอย่างเช่น ภาษา $A$ มีไวยากรณ์ของภาษา
กำหนดไว้ว่าจะต้องประกอบไปด้วยสัญลักษณ์ซึ่งเป็นสมาชิกเฉพาะในเซต 
$\{1,0\}$ เท่านั้นและลำดับของ ``$1$'' จะต้องมาก่อน ``$0$'' ทุกตัว
ดังนั้นถ้ากำหนดให้เซต $B = \{1, 10, 110, 111\}$ 
จะได้ว่าทุกสมาชิกใน $B$ เป็นภาษาที่เป็นซับเซตของภาษา $A$ 
เนื่องจากทุกสมาชิกในเซต $B$ เป็นไปตามไวยากรณ์ของภาษา $A$
}
%
\par{
จากตัวอย่างข้างต้นการอธิบายไวยากรณ์ของภาษาเป็นส่วนสำคัญของโครงสร้างของภาษาเนื่องจากเป็นการตรวจสอบถึงความถูกต้องของโครงสร้างภาษา
วิธีการรูปนัยสำหรับอธิบายไวยากรณ์จึงเป็นสิ่งจำเป็น
เนื่องจากวิธีการรูปนัยสามารถพิสูจน์ความถูกต้องได้โดยหลักการทางคณิตศาสตร์
การนิยามที่ \ref{defn_grammar} เป็นการนิยามไวยากรณ์ของภาษาแบบรูปนัย
%
\begin{defn}
\label{defn_grammar}
ไวยากรณ์ของภาษาจะประกอบไปด้วย 4 ส่วนลำดับ 
$\langle \Sigma, N, P, S\rangle$ โดยที่ 
$\Sigma$ (Terminal symbol) 
เป็นเซตจำกัดของสัญลักษณ์ที่สามารถปรากฎขึ้นได้ในภาษา 
$N$ (Nonterminal symbol) 
คือตัวแสดงลำดับของสัญลักษณ์ที่ยังไม่สมบูรณ์แต่ยังเป็นเพียงส่วนหนึ่งของสมาชิกในภาษาเท่านั้น 
$P$ แสดงกฎการสร้างสำหรับ 
$N$ เพื่อให้ได้มาซึ่งคำที่เป็นสมาชิกของภาษา และ 
$S$ เป็นสมาชิกหนึ่งใน $N$ เป็นตัวบอกจุดเริ่มต้นของการไวยากรณ์
\end{defn}
}
%
\par{
จากตัวอย่างภาษา $A$ ข้างต้น 
จะสามารถอธิบายส่วนของไวยากรณ์ของภาษา
โดยที่ $\Sigma$ คือ $\{0,1\}$ สำหรับ 
$N$ $P$ และ $S$ 
เราจะใช้รูปแบบ 
BNF (Backus Normal Form หรือ Backus–Naur Form)
ในการอธิบาย
}
%

\par{
BNF ถูกใช้ครั้งแรกในการอธิบายไวยากรณ์ของภาษา 
ALGOL 60 โดย 
John Backus และ Peter Naur 
โดยรูปแบบดังกล่าวจะใช้ 
``$\langle \rangle$'' 
สำหรับครอบตัวแสดงลำดับของสัญลักษณ์ที่ยังไม่สมบูรณ์
(เราจะเรียกทับศัพท์ว่าตัว Non-terminal)
ตัวอย่างเช่น
%
\begin{grammar}
<zero at end> ::= <all one>`0' 
\end{grammar}
%
``zero at end'' และ ``all one'' เป็นตัว Non-terminal
``0'' เป็นสัญลักษณ์ที่สามารถปรากฏได้ในภาษา
ส่วนสัญลักษณ์ ((เราจะเรียกทับศัพท์ว่าตัว Terminal) ``::='' 
เป็นส่วนหนึ่งของ BNF ใช้สำหรับอธิบายการเปลี่ยนแปลงของตัว Non-terminal
ซึ่งจะสามารถตีความได้ว่า ตัว Non-terminal ``zero at end'' คือ (หรือประกอบไปด้วย)
``all one'' ตามด้วย ตัว Terminal `0'
}
%
\par{
จะเห็นได้ว่า BNF ก็เป็นภาษาซึ่งมีหลักการในการตีความและโครงสร้างแบบง่าย ๆ 
ไม่ซับซ้อน มีหน้าที่ส่วนใหญ่ใช้ในการอธิบายไวยากรณ์ของภาษาอื่น ๆ 
เราจึงเรียก BNF ว่าเป็น \textit{อภิภาษา} (metalanguage)
}
%
\section{โครงสร้างนามธรรม (Abstract Syntax)}
%
\par{
การกำหนดความสัมพันธ์ขององค์ประกอบในภาษาคอมพิวเตอร์โดยทั่วไปจะสามารถกำหนดได้ในแบบที่ชัดเจน 
(Concrete Syntax) 
ซึ่งการกำหนดแบบชัดเจนจะบอกถึงกระบวนการตรวจสอบความถูกต้องของตัวภาษา 
ตัวอย่างเช่นตัวประโยคดำเนินการบวกและคูณของเลขตั้งแต่ 0 ถึง 9 
ซึ่งกำหนดให้เครื่องหมายคูณดึงตัวแปรไปใช้ก่อนเครื่องหมายบวก 
เราสามารถกำหนดโครงสร้างแบบชัดเจนได้ดังนี้
%
\begin{grammar}
<E> ::= <E> + <T> | <T> 

<T> ::= <T> $\times$ <NUM>  

<NUM> ::= 0 | 1 | 2 | 3 |$\ldots$|9 
\end{grammar}
%
}




%\chapter{ภาษากระบวนการ IMP}
%
\par{
ในเนื้อหาส่วนนี้จะอธิบายโครงสร้างของภาษากระบวนการ
IMP (IMPerative Language) ซึ่งเป็นภาษาอย่างง่าย ๆ 
ที่จะใช้ในการอธิบายเนื้อหาส่วนต่อ ๆ ไป
}
%
\par{
ภาษา IMP มีส่วนประกอบซึ่งสามารถแบ่งแยกออกเป็นลักษณะของเซตได้ตังต่อไปนี้
\begin{itemize}
\item ตัวเลข $N$ จะประกอบไปด้วยตัวเลขทั้งหมด
\item ค่าความจริง $T$ เป็นเซต $\{\mbox{true},\mbox{false}\}$
\item ประโยคดำเนินการทางคณิตศาสตร์ $Aexp$ (Arithmetic expression)
\item ประโยคดำเนินการทางตรรกศาสตร์ $Bexp$ (Boolean expression) 
\item คำสั่ง $Com$ 
\end{itemize}
กำหนดให้ $a \in Aexp$ เราจะเริ่มกำหนดโครงสร้างของ $Aexp$ 
ในรูป BNF ก่อนดังนี้
$$
a ::= a_1 + a_2 | a_1 \times a_2
$$

%\chapter{ความหมายเชิงกระบวนการ}
%
\par{
เนื้อหาในส่วนนี้จะกล่าวถึงการให้ความหมายโดยวิธีการเชิงกระบวนการ 
(Operational Semantics)
ซึ่งเป็นการกำหนดความหมายให้ภาษาคอมพิวเตอร์แบบหนึ่งที่เหมาะสมกับการโปรแกรมเชิงกระบวน
}
%
\par{
หลักการพื้นฐานของการให้ความหมายเชิงกระบวนการ 
คือการใช้เปลี่ยนแปลงของสถานะการทำงานของโปรแกรมเพื่อกำหนดความหมายให้กับโปรแกรม
ฉะนั้นสถานะของโปรแกรม (State) 
จะต้องถูกอธิบายมาอย่างชัดเจนว่าประกอบด้วยอะไรบ้าง 
รวมทั้ง การเปลี่ยนแปลงของสถานะของโปรแกรม (State transition) 
เกิดขึ้นได้อย่างใดบ้าง
}
%
\par{
\begin{example}
%
พิจารณาตัวอย่างของส่วนของโปรแกรมที่อธิบายด้วย BNF 
ต่อไปนี้
\begin{grammar}
<e> ::= `true' | `false' | `not' <e> | `if' <e> <e> <e>
\end{grammar}
\end{example}
}
%
กำหนดให้สถานะของโปรแกรมคือ $Exp$ โดยที่ 
$\langle e \rangle \in Exp$ 
%
\par{
การเปลี่ยนสถานะของโปรแกรม ($\mapsto$)
เป็นความสัมพันธ์ระหว่างสถานะของโปรแกรมก่อนการเปลี่ยนแปลง ($e$) 
กับสถานะของโปรแกรมหลังการเปลี่ยนแปลง ($e'$)
ฉะนั้น การเปลี่ยนสถานะของโปรแกรมสามารถเขียนในรูปคณิตศาสตร์ได้ 
$$
e \mapsto e' \subseteq Exp~\times~Exp
$$
จากโปรแกรมข้างต้นกำหนดให้มีการเปลี่ยนสถานะโปรแกรมทั้งหมด 4 รูปแบบต่อไปนี้
%
\begin{enumerate}
\item \texttt{not} \texttt{true} $\mapsto$ \texttt{false}
\item \texttt{not} \texttt{false} $\mapsto$ \texttt{true}
\item \texttt{if} \texttt{true} $e_1$ $e_2$ $\mapsto$ $e_1$
\item \texttt{if} \texttt{false} $e_1$ $e_2$ $\mapsto$ $e_2$
\end{enumerate}
%
จะเห็นได้ว่านอกจากจะบอกลักษณะของสถานะที่เปลี่ยนแปลงไปของการเปลี่ยนแปลงสถานะแล้ว 
ความหมายยังถูกแสดงออกมาในแต่ละการเปลี่ยนแปลงอีกด้วย ตัวอย่างเช่น
\texttt{if} \texttt{true} $e_1$ $e_2$ $\mapsto$ $e_1$
สามารถอธิบายอยู่ในรูปของความหมายของโปรแกรมได้ว่า 
ถ้าตัวตรวจสอบของการตัดสินใจ 
(\texttt{if true then} $e_1$ \texttt{else} $e_2$)
แบบมีค่าความจริงเป็นจริงแล้ว 
ความหมายของประโยคดังกล่าวจะอยู่เฉพาะในส่วนแรก ($e_1$) เท่านั้น
}
%
\par{
เมื่อมีการกำหนดประโยคสัญลักษณ์ด้วยกระบวนเปลี่ยนแปลงสถานะของโปรแกรมที่ถูกกำหนดมาดังกล่าวข้างต้น
จะทำให้ทราบถึงความหมายของประโยคสัญลักษณ์ดังกล่าวได้ดังเช่นในตัวอย่างต่อไปนี้
\begin{align*}
&~~\texttt{if (not true) (not false) (if true (not true) false)} \\
\mapsto &~~\texttt{if false (not false) (if true (not true) false)} \\
\mapsto &~~\texttt{if true (not true) false} \\
\mapsto &~~\texttt{not true} \\
\mapsto &~~\texttt{false} 
\end{align*}
}

%\chapter{ทดสอบ}
\par{
กำหนดให้ $M = (\{q_0, q_1\}, \{0, 1\}, \{X, Z_0\}, \delta, q_0, Z_0, \emptyset)$ โดยที่ $\delta$ อธิบายดัวต่อไปนี้
\begin{eqnarray*}
\delta(q_0, 0, Z_0) & = & \{(q_0, XZ_0)\} \\
\delta(q_0, 0, X) & = & \{(q_0, XX)\} \\
\delta(q_0, 1, X) & = & \{(q_1, \epsilon)\} \\
\delta(q_1, 1, X) & = & \{(q_1, \epsilon)\} \\
\delta(q_1, \epsilon, X) & = & \{(q_1, \epsilon)\} \\
\delta(q_1, \epsilon, Z_0) & = & \{(q_1, \epsilon)\} \\
\end{eqnarray*}
}

\par{
เริ่มต้นเรากำหนดให้ CFG $G = \{V, T, P, S\}$ (โดยที่ S เป็นตัว non-terminal ตัวแรกของ CFG $G$) ฉะนั้น $V$ จะเป็นเซตดังต่อไปนี้

\begin{eqnarray*}
V & = & \{S , \\
&  & [q_0, X, q_0], [q_0, X, q_1], [q_1, X, q_0], [q_1, X, q_1], \\ 
&  & [q_0, Z_0, q_0], [q_0, Z_0, q_1], [q_1, Z_0, q_0], [q_1, Z_0, q_1] \}
\end{eqnarray*}

เราจะเริ่มต้นที่ $S$ ซึ่งจะเกิดขึ้นจาก state ตั้งต้นของ PDA คือ $q_0$ โดยที่ ค่าบนสุดของ Stack คือ $Z_0$ $(\{q_0, q_1\}, \{0, 1\}, \{X, Z_0\}, \delta, q_0, Z_0, \emptyset)$

\begin{eqnarray*}
S & \rightarrow & [q_0,Z_0, q_0] \\
S & \rightarrow & [q_0,Z_0, q_1] 
\end{eqnarray*}
}

\par{
เราไม่รู้ว่าเมื่อ pop $Z_0$ จาก Stack แล้วจะไปหยุดที่ state ไหน เลยจำเป็นต้องใช้ ทั้ง $[q_0, Z_0, q_0]$ และ $[q_0, Z_0, q_1]$ เราจะเริ่มพิจารณาจาก $[q_0,Z_0, q_0]$
}

\par{
จาก $[q_0,Z_0, q_0]$ เราทราบว่า การทำงานของ PDA จะเริ่มที่ $q_0$ โดยที่ ค่าบนสุดของ Stack คือ $Z_0$ และ เมื่อ $Z_0$ ถูก $pop$ ออกจาก Stack PDA จะไปอยู่ที่ $q_0$ ฉะนั้น จาก $\delta$ ของ PDA เราจะหาการทำงานที่เริ่มจาก $q_0$ และมีค่าบนสุดของ Stack คือ $Z_0$ ซึ่งมีแค่ตัวเดี่ยวคือ $\delta(q_0, 0, Z_0) = \{(q_0, XZ_0)\}$
}

\par{
จากการรูปแบบการทำงานของ PDA $\delta(q_0, 0, Z_0) = \{(q_0, XZ_0)\}$ จะถูกใช้ทำงานเมือมีค่าของ input เป็น 0 ณ $q_0$ โดยค่าบนของ Stack คือ $Z_0$ โดยการทำงานจะเปลี่ยนสถานะเป็น $q_0$ พร้อมทั้งเปลี่ยนค่าบนสุดของ Stack เป็น $XZ_0$ แทน ฉะนั้นเราสามารถอธิบายได้เป็นโครงสร้าง $[qZp]$ ดังต่อไปนี้

\begin{eqnarray*}
[\color{blue}{q_0}, \color{green}{Z_0} \color{black}{, q_0] \rightarrow} \color{red}{0} [  \color{orange}{q_0}
\end{eqnarray*}
}

\par{
ณ state $\color{blue}{q_0}$ และ ค่าบนสุดของ Stack คือ $\color{green}{Z_0}$ ถ้ามี input เป็น  $\color{red}{0}$ จะมีการเปลี่ยน state เป็น $\color{orange}{q_0}$ เนื่องจากการทำงาน PDA ที่คำนึงถึงอยู่ มีการ push $XZ_0$ เข้าไปใน Stack ฉะนั้น

\begin{eqnarray*}
[\color{blue}{q_0}, \color{green}{Z_0}, \color{purple}{q_0}] \rightarrow \color{red}{0} [\color{orange}{q_0}\color{black}{, X, q_x][q_x,Z_0,} \color{purple}{q_0}]
\end{eqnarray*}

โดยที่ $q_x$ เป็น state ใด ๆ ใน PDA เนื่องจาก เราไม่รู้ว่า หลังจากที่ $X$ ถูก pop ออกจาก Stack แล้ว การทำงานของ PDA อยู่ที่ state ไหน ฉะนั้นเราจะได้ CFG production rules ดังต่อไปนี้

\begin{eqnarray*}
S & \rightarrow & [q_0,Z_0, q_0] \\
\left[q_0, Z_0, q_0\right] & \rightarrow & 0 [ q_0, X, q_0][q_0, Z_0, q_0] \\
\left[q_0, Z_0, q_0\right] & \rightarrow & 0 [ q_0, X, q_1][q_1, Z_0, q_0] \\
S & \rightarrow & [q_0,Z_0, q_1] \\
\left[q_0, Z_0, q_1\right] & \rightarrow & 0 [ q_0, X, q_0][q_0, Z_0, q_1] \\
\left[q_0, Z_0, q_1\right] & \rightarrow & 0 [ q_0, X, q_1][q_1, Z_0, q_1]
\end{eqnarray*}
}

\par{
จาก production rule ข้างต้น จะเกิด non-terminal ตัวใหม่สี่ตัวคือ $[ q_0, X, q_0]$
, $[ q_0, X, q_1]$, $[q_1, Z_0, q_0]$ และ $[q_1, Z_0, q_1]$ 
เราจะพิจารณาต่อจาก non-terminal ตัวใหม่เหล่านี้ โดยเริ่มจาก $[q_0, X, q_x]$ ซึ่งจะต้องพิจารณาจาก transition ของ PDA ณ state $q_0$ และ มี $X$ เป็นค่าของจุดสูงสุดของ Stack ซึ่งมีอยู่ด้วยกัน Transition คือ $\delta(q_0, 0, X) = \{(q_0, XX)\}$ และ $\delta(q_0, 1, X) = \{(q_1, \epsilon)\}$
}

\par{
จาก $\delta(q_0, 0, X) = \{(q_0, XX)\}$ เราจะได้

\begin{eqnarray*}
\left[q_0, X, q_0\right] & \rightarrow & 0[q_0, X, q_0][q_0, X, q_0] \\
\left[q_0, X, q_0\right] & \rightarrow & 0[q_0, X, q_1][q_1, X, q_0] \\
\left[q_0, X, q_1\right] & \rightarrow & 0[q_0, X, q_0][q_0, X, q_1]\\
\left[q_0, X, q_1\right] & \rightarrow & 0[q_0, X, q_1][q_1, X, q_1]
\end{eqnarray*}
}

\par{
จาก $\delta(q_0, 1, X) = \{(q_1, \epsilon)\}$ เราจะได้

\begin{eqnarray*}
\left[q_0, X, q_1\right] \rightarrow 1
\end{eqnarray*}
}

\par{
เราจะพิจารณาตัว non-terminal ที่เหลือต่อ ($[q_1, Z_0, q_0]$ และ $[q_1, Z_0, q_1]$) ซึ่งทั้งคู่เป็นกรณีที่ PDA อยู่ ณ state $q_1$ มี ค่าบน Stack คือ $Z_0$
}

\par{
จาก $\delta(q_1, \epsilon, Z_0) = \{(q_1, \epsilon)\}$ เราจะได้

\begin{eqnarray*}
\left[q_1, Z_0, q_1\right] \rightarrow \epsilon
\end{eqnarray*}

ในกรณีของ $[q_1, Z_0, q_0]$ เราไม่สามารถหา Transition ของ PDA มาอธิบายได้ ฉะนั้น $[q_1, Z_0, q_0]$ จะไม่ปรากฎบน production rules ของ CFG แต่เพื่อความสะดวกในการพิจารณาเพื่อตัดออกเราจึงขอใช้ production สมมุติ ดังต่อไปนี้ $[q_1, Z_0, q_0] \rightarrow \emptyset$ แทน production ที่ไม่เกิดขึ้นแทน 
}

\par{
สรุป production rules ของเราในขณะนี้จะแสดงได้ดังต่อไปนี้

\begin{eqnarray*}
S & \rightarrow & [q_0,Z_0, q_0] \\
\left[q_0, Z_0, q_0\right]& \rightarrow & 0 [ q_0, X, q_0][q_0, Z_0, q_0] \\
\left[q_0, Z_0, q_0\right] & \rightarrow & 0 [ q_0, X, q_1][q_1, Z_0, q_0] \\
%
S & \rightarrow & [q_0,Z_0, q_1] \\
\left[q_0, Z_0, q_1\right] & \rightarrow & 0 [ q_0, X, q_0][q_0, Z_0, q_1] \\
\left[q_0, Z_0, q_1\right] & \rightarrow & 0 [ q_0, X, q_1][q_1, Z_0, q_1] \\
%
\left[q_0, X, q_0\right] & \rightarrow & 0[q_0, X, q_0][q_0, X, q_0] \\
\left[q_0, X, q_0\right] & \rightarrow & 0[q_0, X, q_1][q_1, X, q_0] \\
\left[q_0, X, q_1\right] & \rightarrow & 0[q_0, X, q_0][q_0, X, q_1] \\
\left[q_0, X, q_1\right] & \rightarrow & 0[q_0, X, q_1][q_1, X, q_1] \\
%
\left[q_0, X, q_1\right] & \rightarrow & 1 \\
\left[q_1, Z_0, q_1\right] & \rightarrow & \epsilon \\
\left[q_1, Z_0, q_0\right] & \rightarrow & \emptyset
\end{eqnarray*}
}

\par{
เนื่องจาก $[q_1, Z_0, q_0] \rightarrow \emptyset$ เราจึงตัด

\begin{eqnarray*}
S & \rightarrow & [q_0,Z_0, q_0] \\
\left[q_0, Z_0, q_0\right]& \rightarrow & 0 [ q_0, X, q_0][q_0, Z_0, q_0] \\
\color{red}{\left[q_0, Z_0, q_0\right]} & \rightarrow & 0 [ q_0, X, q_1][q_1, Z_0, q_0] \\
%
S & \rightarrow & [q_0,Z_0, q_1] \\
\left[q_0, Z_0, q_1\right] & \rightarrow & 0 [ q_0, X, q_0][q_0, Z_0, q_1] \\
\left[q_0, Z_0, q_1\right] & \rightarrow & 0 [ q_0, X, q_1][q_1, Z_0, q_1] \\
%
\left[q_0, X, q_0\right] & \rightarrow & 0[q_0, X, q_0][q_0, X, q_0] \\
\left[q_0, X, q_0\right] & \rightarrow & 0[q_0, X, q_1][q_1, X, q_0] \\
\left[q_0, X, q_1\right] & \rightarrow & 0[q_0, X, q_0][q_0, X, q_1] \\
\left[q_0, X, q_1\right] & \rightarrow & 0[q_0, X, q_1][q_1, X, q_1] \\
%
\left[q_0, X, q_1\right] & \rightarrow & 1 \\
\left[q_1, Z_0, q_1\right] & \rightarrow & \epsilon \\
\left[q_1, Z_0, q_0\right] & \rightarrow & \emptyset
\end{eqnarray*}

ฉะนั้นเราจะเหลือเพียง

\begin{eqnarray*}
S & \rightarrow & [q_0,Z_0, q_0] \\
\left[q_0, Z_0, q_0\right] & \rightarrow & 0 [ q_0, X, q_0][q_0, Z_0, q_0] \\
%
S & \rightarrow & [q_0,Z_0, q_1] \\
\left[q_0, Z_0, q_1\right] & \rightarrow & 0 [ q_0, X, q_0][q_0, Z_0, q_1] \\
\left[q_0, Z_0, q_1\right] & \rightarrow & 0 [ q_0, X, q_1][q_1, Z_0, q_1] \\
%
\left[q_0, X, q_0\right] & \rightarrow & 0[q_0, X, q_0][q_0, X, q_0] \\
\left[q_0, X, q_0\right] & \rightarrow & 0[q_0, X, q_1][q_1, X, q_0] \\
\left[q_0, X, q_1\right] & \rightarrow & 0[q_0, X, q_0][q_0, X, q_1] \\
\left[q_0, X, q_1\right] & \rightarrow & 0[q_0, X, q_1][q_1, X, q_1] \\
%
\left[q_0, X, q_1\right] & \rightarrow & 1 \\
\left[q_1, Z_0, q_1\right] & \rightarrow & \epsilon 
\end{eqnarray*}
}

\par{
จากข้างต้นยังมี ตัว non-terminal ที่เราต้องหาเพิ่มเติมอีกสองตัวคือ $[q_1, X, q_0]$ และ $[q_1, X, q_1]$
}

\par{
จาก

\begin{eqnarray*}
\delta(q_1, 1, X) & = & \{(q_1, \epsilon)\} \\
\delta(q_1, \epsilon, X) & = & \{(q_1, \epsilon)\}
\end{eqnarray*}

เราจะได้

\begin{eqnarray*}
\left[q_1, X, q_1\right] & \rightarrow & 1 \\
\left[q_1, X, q_1\right] & \rightarrow & \epsilon \\
\left[q_1, X, q_0\right] & \rightarrow & \emptyset \\
\end{eqnarray*}
}

\par{
เพิ่ม production rules ใหม่นี้เข้าไปจะได้

\begin{eqnarray*}
S & \rightarrow & [q_0,Z_0, q_0] \\
\left[q_0, Z_0, q_0\right] & \rightarrow & 0 [ q_0, X, q_0][q_0, Z_0, q_0] \\
%
S & \rightarrow & [q_0,Z_0, q_1] \\
\left[q_0, Z_0, q_1\right] & \rightarrow & 0 [ q_0, X, q_0][q_0, Z_0, q_1] \\
\left[q_0, Z_0, q_1\right] & \rightarrow & 0 [ q_0, X, q_1][q_1, Z_0, q_1] \\
%
\left[q_0, X, q_0\right] & \rightarrow & 0[q_0, X, q_0][q_0, X, q_0] \\
\left[q_0, X, q_0\right] & \rightarrow & 0[q_0, X, q_1][q_1, X, q_0] \\
\left[q_0, X, q_1\right] & \rightarrow & 0[q_0, X, q_0][q_0, X, q_1] \\
\left[q_0, X, q_1\right] & \rightarrow & 0[q_0, X, q_1][q_1, X, q_1] \\
%
\left[q_0, X, q_1\right] & \rightarrow & 1 \\
\left[q_1, Z_0, q_1\right] & \rightarrow & \epsilon \\
\left[q_1, X, q_1\right] & \rightarrow & 1 \\
\left[q_1, X, q_1\right] & \rightarrow & \epsilon \\
\left[q_1, X, q_0\right] & \rightarrow & \emptyset 
\end{eqnarray*}

เนื่องจาก $[q_1, X, q_0] \rightarrow \emptyset$ เราจะต้องตัด

\begin{eqnarray*}
S & \rightarrow & [q_0,Z_0, q_0] \\
\left[q_0, Z_0, q_0\right] & \rightarrow & 0 [ q_0, X, q_0][q_0, Z_0, q_0] \\
%
S & \rightarrow & [q_0,Z_0, q_1] \\
\left[q_0, Z_0, q_1\right] & \rightarrow & 0 [ q_0, X, q_0][q_0, Z_0, q_1] \\
\left[q_0, Z_0, q_1\right] & \rightarrow & 0 [ q_0, X, q_1][q_1, Z_0, q_1] \\
%
\left[q_0, X, q_0\right] & \rightarrow & 0[q_0, X, q_0][q_0, X, q_0] \\
\left[q_0, X, q_0\right] & \rightarrow & 0[q_0, X, q_1]\color{red}{[q_1, X, q_0]} \\
\left[q_0, X, q_1\right] & \rightarrow & 0[q_0, X, q_0][q_0, X, q_1] \\
\left[q_0, X, q_1\right] & \rightarrow & 0[q_0, X, q_1][q_1, X, q_1] \\
%
\left[q_0, X, q_1\right] & \rightarrow & 1 \\
\left[q_1, Z_0, q_1\right] & \rightarrow & \epsilon \\
\left[q_1, X, q_1\right] & \rightarrow & 1 \\
\left[q_1, X, q_1\right] & \rightarrow & \epsilon
\end{eqnarray*}

เราจะได้

\begin{eqnarray*}
S & \rightarrow & [q_0,Z_0, q_0] \\
\left[q_0, Z_0, q_0\right] & \rightarrow & 0 [ q_0, X, q_0][q_0, Z_0, q_0] \\
%
S & \rightarrow & [q_0,Z_0, q_1] \\
\left[q_0, Z_0, q_1\right] & \rightarrow & 0 [ q_0, X, q_0][q_0, Z_0, q_1] \\
\left[q_0, Z_0, q_1\right] & \rightarrow & 0 [ q_0, X, q_1][q_1, Z_0, q_1] \\
%
\left[q_0, X, q_0\right] & \rightarrow & 0[q_0, X, q_0][q_0, X, q_0] \\
\left[q_0, X, q_1\right] & \rightarrow & 0[q_0, X, q_0][q_0, X, q_1] \\
\left[q_0, X, q_1\right] & \rightarrow & 0[q_0, X, q_1][q_1, X, q_1] \\
%
\left[q_0, X, q_1\right] & \rightarrow & 1 \\
\left[q_1, Z_0, q_1\right] & \rightarrow & \epsilon \\
\left[q_1, X, q_1\right] & \rightarrow & 1 \\
\left[q_1, X, q_1\right] & \rightarrow & \epsilon
\end{eqnarray*}
}

\par{
เราจะเร่ิมพิจารณา production ที่ไม่สามารถทำให้เกิด terminal ขึ้นได้โดยเริ่มจาก ทำเครื่องหมายตัวที่ทำให้เกิด terminal ก่อน ดังต่อไปนี้
}

\par{
เริ่มต้นจาก การเปลี่ยนเป็น terminal ได้โดยตรง

\begin{eqnarray*}
S & \rightarrow & [q_0,Z_0, q_0] \\
\left[q_0, Z_0, q_0\right] & \rightarrow & 0 [ q_0, X, q_0][q_0, Z_0, q_0] \\
%
S & \rightarrow & [q_0,Z_0, q_1] \\
\left[q_0, Z_0, q_1\right] & \rightarrow & 0 [ q_0, X, q_0][q_0, Z_0, q_1] \\
\left[q_0, Z_0, q_1\right] & \rightarrow & 0 [ q_0, X, q_1][q_1, Z_0, q_1] \\
%
\left[q_0, X, q_0\right] & \rightarrow & 0[q_0, X, q_0][q_0, X, q_0] \\
\left[q_0, X, q_1\right] & \rightarrow & 0[q_0, X, q_0][q_0, X, q_1] \\
\left[q_0, X, q_1\right] & \rightarrow & 0[q_0, X, q_1][q_1, X, q_1] \\
%
\color{blue}{\left[q_0, X, q_1\right]} & \rightarrow & 1 \\
\color{blue}{\left[q_1, Z_0, q_1\right]} & \rightarrow & \epsilon \\
\color{blue}{\left[q_1, X, q_1\right]} & \rightarrow & 1 \\
\color{blue}{\left[q_1, X, q_1\right]} & \rightarrow & \epsilon
\end{eqnarray*}
}

\par{
จากนั้นทำเครื่องหมายต่อใน production ต่าง ๆ ที่มีตัวที่ถูกทำเครื่องหมายแล้วดังต่อไปนี้

\begin{eqnarray*}
S & \rightarrow & [q_0,Z_0, q_0] \\
\left[q_0, Z_0, q_0\right] & \rightarrow & 0 [ q_0, X, q_0][q_0, Z_0, q_0] \\
%
S & \rightarrow & [q_0,Z_0, q_1] \\
\left[q_0, Z_0, q_1\right] & \rightarrow & 0 [ q_0, X, q_0][q_0, Z_0, q_1] \\
\left[q_0, Z_0, q_1\right] & \rightarrow & 0 \color{blue}{[ q_0, X, q_1]}\color{blue}{[q_1, Z_0, q_1]} \\
%
\left[q_0, X, q_0\right] & \rightarrow & 0[q_0, X, q_0][q_0, X, q_0] \\
\left[q_0, X, q_1\right] & \rightarrow & 0[q_0, X, q_0]\color{blue}{[q_0, X, q_1]} \\
\left[q_0, X, q_1\right] & \rightarrow & 0\color{blue}{[q_0, X, q_1]}\color{blue}{[q_1, X, q_1]} \\
%
\color{blue}{\left[q_0, X, q_1\right]} & \rightarrow & 1 \\
\color{blue}{\left[q_1, Z_0, q_1\right]} & \rightarrow & \epsilon \\
\color{blue}{\left[q_1, X, q_1\right]} & \rightarrow & 1 \\
\color{blue}{\left[q_1, X, q_1\right]} & \rightarrow & \epsilon
\end{eqnarray*}
}

\par{
ถ้า non-terminal ทุกตัวทางด้านขวามือของ production rule สามารถเปลี่ยนเป็น terminal ได้ทั้งหมดแสดงว่า ด้านซ้ายมือของ production rule สามารถสร้าง terminal ขึ้นได้ ฉะนั้น

\begin{eqnarray*}
S & \rightarrow & [q_0,Z_0, q_0] \\
\left[q_0, Z_0, q_0\right] & \rightarrow & 0 [ q_0, X, q_0][q_0, Z_0, q_0] \\
%
S & \rightarrow & [q_0,Z_0, q_1] \\
\left[q_0, Z_0, q_1\right] & \rightarrow & 0 [ q_0, X, q_0][q_0, Z_0, q_1] \\
\color{red}{\left[q_0, Z_0, q_1\right]} & \rightarrow & 0 \color{blue}{[ q_0, X, q_1]}\color{blue}{[q_1, Z_0, q_1]} \\
%
\left[q_0, X, q_0\right] & \rightarrow & 0[q_0, X, q_0][q_0, X, q_0] \\
\left[q_0, X, q_1\right] & \rightarrow & 0[q_0, X, q_0]\color{blue}{[q_0, X, q_1]} \\
\color{red}{\left[q_0, X, q_1\right]} & \rightarrow & 0\color{blue}{[q_0, X, q_1]}\color{blue}{[q_1, X, q_1]} \\
%
\color{blue}{\left[q_0, X, q_1\right]} & \rightarrow & 1 \\
\color{blue}{\left[q_1, Z_0, q_1\right]} & \rightarrow & \epsilon \\
\color{blue}{\left[q_1, X, q_1\right]} & \rightarrow & 1 \\
\color{blue}{\left[q_1, X, q_1\right]} & \rightarrow & \epsilon 
\end{eqnarray*}
}

\par{
ทำกระบวนการข้างต้นจนกว่าจะไม่มีการทำเครื่องหมายเพิ่มเติมจึงจะหยุด เราจะได้ production rule ที่มีการทำเครื่องหมายแล้วดังต่อไปนี้

\begin{eqnarray*}
S & \rightarrow & [q_0,Z_0, q_0] \\
\left[q_0, Z_0, q_0\right] & \rightarrow & 0 [ q_0, X, q_0][q_0, Z_0, q_0] \\
%
\color{blue}{S} & \rightarrow & \color{blue}{[q_0,Z_0, q_1]} \\
\left[q_0, Z_0, q_1\right] & \rightarrow & 0 [ q_0, X, q_0]\color{blue}{[q_0, Z_0, q_1]} \\
\color{blue}{\left[q_0, Z_0, q_1\right]} & \rightarrow & 0 \color{blue}{[ q_0, X, q_1]}\color{blue}{[q_1, Z_0, q_1]} \\
%
\left[q_0, X, q_0\right] & \rightarrow & 0[q_0, X, q_0][q_0, X, q_0] \\
\left[q_0, X, q_1\right] & \rightarrow & 0[q_0, X, q_0]\color{blue}{[q_0, X, q_1]} \\
\color{blue}{\left[q_0, X, q_1\right]} & \rightarrow & 0\color{blue}{[q_0, X, q_1]}\color{blue}{[q_1, X, q_1]} \\
%
\color{blue}{\left[q_0, X, q_1\right]} & \rightarrow & 1 \\
\color{blue}{\left[q_1, Z_0, q_1\right]} & \rightarrow & \epsilon \\
\color{blue}{\left[q_1, X, q_1\right]} & \rightarrow & 1 \\
\color{blue}{\left[q_1, X, q_1\right]} & \rightarrow & \epsilon 
\end{eqnarray*}
}

\par{
จาก production rules ที่ทำเครื่องหมายไว้แล้ว เราจะตัด production rules ทั้งหมดที่ตัวด้านซ้ายของ production rule ไม่ถูกทำเครื่องหมายไว้ เนื่องจาก production เหล่านั้นไม่สามารถทำให้เกิด terminal ขึ้นมาได้ ฉะนั้นเราจะได้ production rules ทั้งหมดดังต่อไปนี้

\begin{eqnarray*}
\color{blue}{S} & \rightarrow & \color{blue}{[q_0,Z_0, q_1]} \\
\color{blue}{\left[q_0, Z_0, q_1\right]} & \rightarrow & 0 \color{blue}{[ q_0, X, q_1]}\color{blue}{[q_1, Z_0, q_1]} \\
%
\color{blue}{\left[q_0, X, q_1\right]} & \rightarrow & 0\color{blue}{[q_0, X, q_1]}\color{blue}{[q_1, X, q_1]} \\
%
\color{blue}{\left[q_0, X, q_1\right]} & \rightarrow & 1 \\
\color{blue}{\left[q_1, Z_0, q_1\right]} & \rightarrow & \epsilon \\
\color{blue}{\left[q_1, X, q_1\right]} & \rightarrow & 1 \\
\color{blue}{\left[q_1, X, q_1\right]} & \rightarrow & \epsilon \\
\end{eqnarray*}
}

\par{
ตัวอย่างข้อต่อมา กำหนดให้ CFG $G_1$ ดังต่อไปนี้ 

\begin{eqnarray*}
S & \rightarrow & aAA \\
A & \rightarrow & aS|bS|a 
\end{eqnarray*}

ให้สร้าง PDA ที่มีการทำงานเหมือนกันกับ CFG $G_1$
}

\par{
เราสามารถสร้าง PDA $M = (\{q_0, q_1\},\{a, b\}, \{a, b, A, S, Z_0\}, \delta, q_0,  Z_0, \emptyset)$

โดยที่มี $\delta$ ดังต่อไปนี้ 

\begin{eqnarray*}
\delta(q_0, \epsilon, Z_0) & = & \{(q_1, SZ_0)\} \\
\delta(q_1, \epsilon, S) & = & \{(q_1, aAA)\} \\
\delta(q_1, \epsilon, A) & = & \{(q_1, aS), (q_1, bS), (q_1, a)\} \\
\delta(q_1, a, a) & = & \{(q_1, \epsilon)\} \\ 
\delta(q_1, b, b) & = & \{(q_1, \epsilon)\} \\
\delta(q_1, \epsilon, Z_0) & = & \{(q_1, \epsilon)\}
\end{eqnarray*}
}



\par{
Given a grammar for the language recognized by $M=(\{q_0, q_1\} ...$
}

\begin{eqnarray*}
S & \rightarrow & [q_0, Z_0, q_0] \\
S & \rightarrow & [q_0, Z_0, q_1] 
\end{eqnarray*}

From $\delta(q_0, 1, Z_0) = \{(q_0, XZ_0)\}$

\begin{eqnarray*}
\left[q_0, Z_0, q_0\right] & \rightarrow 1[q_0, X, q_0][q_0, Z_0, q_0] \\
\left[q_0, Z_0, q_0\right] & \rightarrow 1[q_0, X, q_1][q_1, Z_0, q_0] \\
\left[q_0, Z_0, q_1\right] & \rightarrow 1[q_0, X, q_0][q_0, Z_0, q_1] \\
\left[q_0, Z_0, q_1\right] & \rightarrow 1[q_0, X, q_1][q_1, Z_0, q_1] 
\end{eqnarray*}

From $\delta(q_0, \epsilon, Z_0) = \{(q_0, \epsilon)\}$

\begin{eqnarray*}
\left[q_0, Z_0, q_0\right] & \rightarrow & \epsilon 
\end{eqnarray*}

We got

\begin{eqnarray*}
S & \rightarrow & [q_0, Z_0, q_0] \\
S & \rightarrow & [q_0, Z_0, q_1] \\
\left[q_0, Z_0, q_0\right] & \rightarrow & 1[q_0, X, q_0][q_0, Z_0, q_0] \\
\left[q_0, Z_0, q_0\right] & \rightarrow & 1[q_0, X, q_1][q_1, Z_0, q_0] \\
\left[q_0, Z_0, q_1\right] & \rightarrow & 1[q_0, X, q_0][q_0, Z_0, q_1] \\
\left[q_0, Z_0, q_1\right] & \rightarrow & 1[q_0, X, q_1][q_1, Z_0, q_1] \\
\left[q_0, Z_0, q_0\right] & \rightarrow & \epsilon 
\end{eqnarray*}

From $\delta(q_0, 0, X) = \{(q_1, X)\}$ and $\delta(q_0, 1, X) = \{(q_0, XX)\}$

\begin{eqnarray*}
\left[q_0, X, q_0\right] & \rightarrow & 0[q_1, X, q_0] \\
\left[q_0, X, q_1\right] & \rightarrow & 0[q_1, X, q_1] \\
\left[q_0, X, q_0\right] & \rightarrow & 1[q_0, X, q_0][q_0, X, q_0] \\
\left[q_0, X, q_0\right] & \rightarrow & 1[q_0, X, q_1][q_1, X, q_0] \\
\left[q_0, X, q_1\right] & \rightarrow & 1[q_0, X, q_0][q_0, X, q_0] \\
\left[q_0, X, q_1\right] & \rightarrow & 1[q_0, X, q_1][q_1, X, q_0] 
\end{eqnarray*}

Currently, we have the following production rules

\begin{eqnarray*}
S & \rightarrow & [q_0, Z_0, q_0] \\
S & \rightarrow & [q_0, Z_0, q_1] \\
\left[q_0, Z_0, q_0\right] & \rightarrow & 1[q_0, X, q_0][q_0, Z_0, q_0] \\
\left[q_0, Z_0, q_0\right] & \rightarrow & 1[q_0, X, q_1][q_1, Z_0, q_0] \\
\left[q_0, Z_0, q_1\right] & \rightarrow & 1[q_0, X, q_0][q_0, Z_0, q_1] \\
\left[q_0, Z_0, q_1\right] & \rightarrow & 1[q_0, X, q_1][q_1, Z_0, q_1] \\
\left[q_0, Z_0, q_0\right] & \rightarrow & \epsilon \\
\left[q_0, X, q_0\right] & \rightarrow & 0[q_1, X, q_0] \\
\left[q_0, X, q_1\right] & \rightarrow & 0[q_1, X, q_1] \\
\left[q_0, X, q_0\right] & \rightarrow & 1[q_0, X, q_0][q_0, X, q_0] \\
\left[q_0, X, q_0\right] & \rightarrow & 1[q_0, X, q_1][q_1, X, q_0] \\
\left[q_0, X, q_1\right] & \rightarrow & 1[q_0, X, q_0][q_0, X, q_0] \\
\left[q_0, X, q_1\right] & \rightarrow & 1[q_0, X, q_1][q_1, X, q_0] 
\end{eqnarray*}



%\input{np_completeness}
%\input{steiner_tree}

\bibliographystyle{apalike}
\bibliography{ref}
\printindex

\end{document}