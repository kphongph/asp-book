\chapter{บทนำ}
\par{
Answer Set Programming (ASP) \index{Answer Set Programming}
เป็นภาษาโปรแกรมเชิงประกาศ 
(Declarative Programming Language)
\index{Declarative Programming Language}
มีลักษณะการเขียนแตกต่างจะภาษาโปรแกรมเชิงกระบวนการ 
(Imperative Programming) 
\index{Imperative Programming Language}
ซึ่งจะเน้นเรื่องลำดับขั้นตอนการทำงานของเครื่องจักร 
ตัวอย่างของภาษาเชิงกระบวนการคือ ภาษาจาวา ภาษาซี เป็นต้น
ในขณะที่ภาษาโปรแกรมเชิงกระบวนการเน้นขั้นตอนการทำงาน 
การภาษาโปรแกรมเชิงประกาศนั้นจะเน้นที่จะอธิบายว่าปัญหาคืออะไร 
แล้วให้เครื่องจักรทำงานแก้ปัญหาดังกล่าวเองโดยไม่จำเป็นต้องบอกให้เครื่องจักรทำงานอย่างไรบ้าง
เพื่อเข้าใจในความแตกต่างของการพัฒนาภาษาทั้งสองรูปแบบ 
พิจารณาจากตัวอย่างปัญหา \ref{ex_waterjugproblem}
}
%

\begin{example}
\label{ex_waterjugproblem}
ปัญหาเหยือกน้ำ (Water Jug Problem) 
\index{Water Jug Problem} กำหนดให้มีเหยือกน้ำ 2 ใบ
มีขนาดความจุ 4 ลิตร กับ 3 ลิตร ตามลำดับ
โดยที่เหยือกน้ำทั้งสองใบไม่มีขีดสำหรับวัดความจุของน้ำ
เป็นไปได้หรือไม่ว่าจะให้มีน้ำเหลืออยู่จำนวน 2 ลิตรในเหยือกขนาด 4 ลิตร
ถ้ากำหนดให้มีน้ำสำหรับใส่ในแต่ละเหยือกได้ตลอด
\end{example}
%
\par{
เราจะเริ่มต้นด้วยการใช้ภาษาเชิงกระบวนการ 
ในการแก้ปัญหาจากตัวอย่างปัญหา \ref{ex_waterjugproblem} 
อันดับแรกในการใช้ภาษาเชิงกระบวนการ 
คือการอธิบายสถานะการทำงานของโปรแกรม 
ซึ่งสถานะจะต้องรวมตัวแปรทุกตัวที่มีผลต่อการทำงานของโปรแกรม
จากปัญหาดังกล่าวสถานะของการทำงานจะสามารถอธิบายได้ด้วยจำนวนน้ำที่อยู่ในเหยือกน้ำแต่ละใบ
ฉะนั้นกำหนดให้ $w_3$ และ $w_4$ 
เป็นปริมาณน้ำที่อยู่ในเหยือกขนาด 3 และ 4 ลิตรตามลำดับ
สถานะการทำงานของโปรแกรม ณ เวลา $t$ ใด ๆ 
สามารถอธิยายได้ด้วยปริมาณของน้ำในเหยือกทั้งสองใบในลักษณะเป็นลำดับดังนี้ 
$(w_3, w_4)_t$
}
%
\par{
เมื่อได้สถานะในการอธิบายลำดับการทำงานของโปรแกรมแล้ว 
เราจะเริ่มกำหนดหรืออธิบายการเปลี่ยนสถานะของการทำงาน
จากปัญหาดังกล่าว 
ถ้ากำหนดให้สถานะปัจจุบันคือ $(w_3, w_4)_t$ 
การเปลี่ยนสถานะหรือการเปลี่ยนแปลงปริมาณน้ำในแต่ละใบจะเกิดขึ้นได้สามแบบ
คือ (1) การเติมน้ำให้เต็มสำหรับเหยือกที่ยังมีน้ำไม่เต็ม กับ 
(2) การเทน้ำจากเหยือกใบหนึ่งไปยังอีกใบหนึ่ง และ (3) การเทน้ำในเหยือกทิ้ง
กรณีที่เป็นไปได้ทั้งหมดของ $(w_3',w_4')_{t+1}$ 
สามารถแสดงได้ดังต่อไปนี้
\begin{enumerate}
\item การเพิ่มน้ำจากแหล่งน้ำให้กับเหยือกน้ำใบใดใบหนึ่งจนเต็ม 
\begin{eqnarray}
(w_3',w_4') \leftarrow (3, w_4) \label{eq_wj3} \\
(w_3',w_4') \leftarrow (w_3, 4) \label{eq_wj4}
\end{eqnarray}
\item การถ่ายเทน้ำจากเหยือก 4 ลิตรไปยังเหยือก 3 ลิตร
\begin{itemize}
\item ถ้า $w_3 + w_4 \le 3$ 
\begin{eqnarray}
(w_3',w_4') \leftarrow (w_3+w_4, 0) \label{eq_wj4to3_1}
\end{eqnarray}
\item ถ้า $w_3 + w_4 > 3$ 
\begin{eqnarray}
(w_3',w_4') \leftarrow (3, w_4-(3-w_3)) \label{eq_wj4to3_2}
\end{eqnarray}
\end{itemize}
\item การถ่ายเทน้ำจากเหยือก 3 ลิตรไปยังเหยือก 4 ลิตร
\begin{itemize}
\item ถ้า $w_3 + w_4 \le 4$ 
\begin{eqnarray}
(w_3',w_4') \leftarrow (0, w_3+w_4) \label{eq_wj3to4_1}
\end{eqnarray}
\item ถ้า $w_3 + w_4 > 4$ 
\begin{eqnarray}
(w_3',w_4') \leftarrow (w_3-(4-w_4),4) \label{eq_wj3to4_2}
\end{eqnarray}
\end{itemize}
\item การเทน้ำทั้งหมดในเหยือกทิ้ง
\begin{eqnarray}
(w_3',w_4') \leftarrow (0, w_4) \label{eq_wje3} \\
(w_3',w_4') \leftarrow (w_3, 0) \label{eq_wje4}
\end{eqnarray}
\end{enumerate}
%
ถ้ากำหนดให้ $(w_3,w_4)_t = (0,0)_0$ จากการเปลี่ยนสถานะข้างต้น $(w_3',w_4')_{t+1}$ 
จะสามารถเป็นหนึ่งในสมาชิกของเซตต่อไปนี้ $\{(0,0)_1,(3,0)_1,(0,4)_1\}$ ซึ่ง 
$(0,0)_1$ เกิดขึ้นได้จากหนึ่งในกฏที่อธิบายด้วยสมการต่อไปนี้ สมการที่ (\ref{eq_wj3to4_1})  
สมการที่ (\ref{eq_wj4to3_1}) สมการที่ (\ref{eq_wje3}) และ 
สมการที่ (\ref{eq_wje4})
ส่วน $(3,0)_1$ และ $(0,4)_1$ เกิดขึ้นจาก 
สมการที่ (\ref{eq_wj3}) และ สมการที่ (\ref{eq_wj4}) ตามลำดับ
}
%
\par{
จะเห็นได้ว่าเมื่อพิจารณาเฉพาะการเปลี่ยนสถานะของปริมาณน้ำ สถานะที่ $(0,0)_0$ กับ สถานะที่ $(0,0)_1$
ไม่มีข้อแตกต่างกันเนื่องจากทั้งสองมีปริมาณน้ำในแต่ละเหยือกเท่ากัน  
ฉะนั้นเราจะไม่คิดเรื่องของเวลาสำหรับระบุลำดับของการทำงานของโปรแกรม 
(สาเหตุที่ผู้แต่งใช้ $t$ ก่อนหน้านี้เพื่อที่จะอธิบายลำดับของการเปลี่ยนแปลงสถานะของการทำงานของโปรแกรม)
}
%
\par{
จากการกำหนดรูปแบบสถานะเพื่ออธิบายการทำงานของโปรแกรมและลักษณะการเปลี่ยนแปลงของแต่ละสถานะ
เราสามารถสร้างการกราฟแสดงการเปลี่ยนแปลงของสถานะทั้งหมดที่เป็นไปได้
ดังแสดงในรูปภาพที่ \ref{fg_allstatewj}
รูปภาพที่ \ref{fg_statewj} แสดงการเปลี่ยนแปลงสถานะจาก $(0,0)$ ไปยัง $(3,2)$ 
ซึ่งเป็นตอบปัญหาว่าสามารถมีน้ำจำนวน 2 ลิตรเหลือให้เหยือกขนาด 4 ลิตรได้
}
%
%\begin{figure}[t]
%\centering
% \pstree[treemode=R,levelsep=1.8,radius=2pt]{\Toval{(0,0)}}{
%   \pstree{\Toval{(3,0)}}{\Toval{(3,4)}\Toval{(0,3)}}
%   \pstree{\Toval{(0,4)}}{
%     \pstree{\Toval{(3,1)}}{
%       \pstree{\Toval{(0,1)}}{
%         \pstree{\Toval{(1,0)}}{
%           \pstree{\Toval{(1,4)}}{\Toval{(3,2)}}
%         }
%       }
%     }
%   }
% }
%\caption{แสดงการเปลี่ยนสถานะของปริมาณน้ำโดยเริ่มต้นจากที่มีน้ำ 0 
%ลิตรบรรจุอยู่ในเหยือกทั้งสองใบ ไปยังคำตอบที่ต้องการคือมีน้ำเหลืออยู่ 2
%ลิตรในเหยือกขนาด 4 ลิตร
%โดยที่เราจะพิจารณาเฉพาะสถานะที่ไม่ซ้ำกันเท่านั้น 
%ตัวอย่างเช่น จาก $(0,4)$ สามารถเปลี่ยนเป็น $(0,0)$ หรือ $(3,4)$ ได้
%แต่เนื่องจากทั้งสองสถานะมีอยู่แล้วจึงไม่มีความจำเป็นต้องสร้างสถานะดังกล่าวใหม่}
%\label{fg_statewj}
%\end{figure}
%
\begin{figure}[t]
\centering
\begin{pspicture}(0,-0.5)(6,5)
\rput(0,4.5){\rnode{c00}{$(0,0)$}}
\rput(1.5,4.5){\rnode{c01}{$(0,1)$}}
\rput(3,4.5){\rnode{c02}{$(0,2)$}}
\rput(4.5,4.5){\rnode{c03}{$(0,3)$}}
\rput(6,4.5){\rnode{c04}{$(0,4)$}}
%
\rput(0,3){\rnode{c10}{$(1,0)$}}
\rput(1.5,3){\rnode{c11}{$(1,1)$}}
\rput(3,3){\rnode{c12}{$(1,2)$}}
\rput(4.5,3){\rnode{c13}{$(1,3)$}}
\rput(6,3){\rnode{c14}{$(1,4)$}}
%
\rput(0,1.5){\rnode{c20}{$(2,0)$}}
\rput(1.5,1.5){\rnode{c21}{$(2,1)$}}
\rput(3,1.5){\rnode{c22}{$(2,2)$}}
\rput(4.5,1.5){\rnode{c23}{$(2,3)$}}
\rput(6,1.5){\rnode{c24}{$(2,4)$}}
%
\rput(0,0){\rnode{c30}{$(3,0)$}}
\rput(1.5,0){\rnode{c31}{$(3,1)$}}
\rput(3,0){\rnode{c32}{$(3,2)$}}
\rput(4.5,0){\rnode{c33}{$(3,3)$}}
\rput(6,0){\rnode{c34}{$(3,4)$}}
%
\end{pspicture}
\caption{แสดงการเปลี่ยนสถานะของปริมาณน้ำโดยเริ่มต้นจากที่มีน้ำ 0 
ลิตรบรรจุอยู่ในเหยือกทั้งสองใบ ไปยังคำตอบที่ต้องการคือมีน้ำเหลืออยู่ 2
ลิตรในเหยือกขนาด 4 ลิตร
โดยที่เราจะพิจารณาเฉพาะสถานะที่ไม่ซ้ำกันเท่านั้น 
ตัวอย่างเช่น จาก $(0,4)$ สามารถเปลี่ยนเป็น $(0,0)$ หรือ $(3,4)$ ได้
แต่เนื่องจากทั้งสองสถานะมีอยู่แล้วจึงไม่มีความจำเป็นต้องสร้างสถานะดังกล่าวใหม่}
\label{fg_allstatewj}
\end{figure}
%
\begin{figure}[t]
\centering
\begin{pspicture}(0,-0.5)(6,5)
\rput(0,4.5){\rnode{c00}{$(0,0)$}}
\rput(1.5,4.5){\rnode{c01}{$(0,1)$}}
\rput(3,4.5){\rnode{c02}{$(0,2)$}}
\rput(4.5,4.5){\rnode{c03}{$(0,3)$}}
\rput(6,4.5){\rnode{c04}{$(0,4)$}}
%
\rput(0,3){\rnode{c10}{$(1,0)$}}
%\rput(1.5,3){\rnode{c11}{$(1,1)$}}
%\rput(3,3){\rnode{c12}{$(1,2)$}}
%\rput(4.5,3){\rnode{c13}{$(1,3)$}}
\rput(6,3){\rnode{c14}{$(1,4)$}}
%
\rput(0,1.5){\rnode{c20}{$(2,0)$}}
%\rput(1.5,1.5){\rnode{c21}{$(2,1)$}}
%\rput(3,1.5){\rnode{c22}{$(2,2)$}}
%\rput(4.5,1.5){\rnode{c23}{$(2,3)$}}
\rput(6,1.5){\rnode{c24}{$(2,4)$}}
%
\rput(0,0){\rnode{c30}{$(3,0)$}}
\rput(1.5,0){\rnode{c31}{$(3,1)$}}
\rput(3,0){\rnode{c32}{$(3,2)$}}
\rput(4.5,0){\rnode{c33}{$(3,3)$}}
\rput(6,0){\rnode{c34}{$(3,4)$}}
%
\psset{nodesep=3pt,arrowsize=3pt 3}
\ncarc[arcangleB=30,arcangleA=30,linestyle=solid]{->}{c00}{c04}\mput*{1}
\ncline[linestyle=solid]{->}{c04}{c31}\mput*{2}
\ncline[linestyle=solid]{->}{c31}{c01}\mput*{3}
\ncline[linestyle=solid]{->}{c01}{c10}\mput*{4}
\ncline[linestyle=solid]{->}{c10}{c14}\mput*{5}
\ncline[linestyle=solid]{->}{c14}{c32}\mput*{6}
\end{pspicture}
\caption{แสดงการเปลี่ยนสถานะของปริมาณน้ำโดยเริ่มต้นจากที่มีน้ำ 0 
ลิตรบรรจุอยู่ในเหยือกทั้งสองใบ ไปยังคำตอบที่ต้องการคือมีน้ำเหลืออยู่ 2
ลิตรในเหยือกขนาด 4 ลิตร
โดยที่เราจะพิจารณาเฉพาะสถานะที่ไม่ซ้ำกันเท่านั้น 
ตัวอย่างเช่น จาก $(0,4)$ สามารถเปลี่ยนเป็น $(0,0)$ หรือ $(3,4)$ ได้
แต่เนื่องจากทั้งสองสถานะมีอยู่แล้วจึงไม่มีความจำเป็นต้องสร้างสถานะดังกล่าวใหม่}
\label{fg_allstatewj}
\end{figure}
%
%
\begin{figure}[t]
\centering
\begin{pspicture}(0,-0.5)(6,5)
\rput(0,4.5){\rnode{c00}{$(0,0)$}}
\rput(1.5,4.5){\rnode{c01}{$(0,1)$}}
\rput(3,4.5){\rnode{c02}{$(0,2)$}}
\rput(4.5,4.5){\rnode{c03}{$(0,3)$}}
\rput(6,4.5){\rnode{c04}{$(0,4)$}}
%
\rput(0,3){\rnode{c10}{$(1,0)$}}
%\rput(1.5,3){\rnode{c11}{$(1,1)$}}
%\rput(3,3){\rnode{c12}{$(1,2)$}}
%\rput(4.5,3){\rnode{c13}{$(1,3)$}}
\rput(6,3){\rnode{c14}{$(1,4)$}}
%
\rput(0,1.5){\rnode{c20}{$(2,0)$}}
%\rput(1.5,1.5){\rnode{c21}{$(2,1)$}}
%\rput(3,1.5){\rnode{c22}{$(2,2)$}}
%\rput(4.5,1.5){\rnode{c23}{$(2,3)$}}
\rput(6,1.5){\rnode{c24}{$(2,4)$}}
%
\rput(0,0){\rnode{c30}{$(3,0)$}}
\rput(1.5,0){\rnode{c31}{$(3,1)$}}
\rput(3,0){\rnode{c32}{$(3,2)$}}
\rput(4.5,0){\rnode{c33}{$(3,3)$}}
\rput(6,0){\rnode{c34}{$(3,4)$}}
%
\psset{nodesep=3pt,arrowsize=3pt 3}
\ncarc[arcangleB=-35,arcangleA=-35,linestyle=solid]{->}{c00}{c30}\mput*{1}
\ncline[linestyle=solid]{->}{c30}{c03}\mput*{2}
\ncline[linestyle=solid]{->}{c03}{c33}\mput*{3}
\ncline[linestyle=solid]{->}{c33}{c24}\mput*{4}
\ncline[linestyle=solid]{->}{c24}{c20}\mput*{5}
\ncline[linestyle=solid]{->}{c20}{c02}\mput*{6}
\end{pspicture}
\caption{แสดงการเปลี่ยนสถานะของปริมาณน้ำโดยเริ่มต้นจากที่มีน้ำ 0 
ลิตรบรรจุอยู่ในเหยือกทั้งสองใบ ไปยังคำตอบที่ต้องการคือมีน้ำเหลืออยู่ 2
ลิตรในเหยือกขนาด 4 ลิตร
โดยที่เราจะพิจารณาเฉพาะสถานะที่ไม่ซ้ำกันเท่านั้น 
ตัวอย่างเช่น จาก $(0,4)$ สามารถเปลี่ยนเป็น $(0,0)$ หรือ $(3,4)$ ได้
แต่เนื่องจากทั้งสองสถานะมีอยู่แล้วจึงไม่มีความจำเป็นต้องสร้างสถานะดังกล่าวใหม่}
\label{fg_allstatewj}
\end{figure}
%
%
%
\par{
จากรูปภาพที่ \ref{fg_statewj} 
ลักษณะการทำงานของโปรแกรมแบบเชิงกระบวนการจะเริ่มต้นด้วยการกำหนดค่าเริ่มต้นของสถานะ 
(ในที่นี่คือ $(0,0)$) 
จากนั้นจะทำการสร้างสถานะที่เป็นไปได้ทั้งหมดแล้วตรวจสอบว่า
สถานะที่ถูกสร้างขึ้นเป็นสถานะที่เป็นคำตอบหรือไม่ 
(ในที่นี่คือ $(w_3,2)$ โดยที่ $w_3$ เป็นค่าตัวเลขใด ๆ 
ที่มากกว่าหรือเท่ากับ 0 แต่น้อยกว่าหรือเท่ากับ 3)
ถ้าสถานะด้งกล่าวไม่ใช่สถานะที่เป็นคำตอบ สถานะดังกล่าวเคยถูกพิจารณา 
หรือรอการพิจารณาหรือไม่
ถ้ายังจะเพิ่มสถานะดังกล่าวในเซตของสถานะที่รอการพิจารณา
}
%
\par{
กำหนดให้ $W$ เป็นเซตของสถานะที่รอการพิจารณา $C$ เป็นเซตของสถานะที่พิจารณาเรียบร้อยแล้ว
ตัวอย่างของการโปรแกรมแบบเชิงกระบวนการสามารถเขียนได้ดังแสดงใน 
การแสดงขั้นตอนวิธีการ (Algorithm) \ref{al_wj}
โดยที่ $N$ เป็นเซตของสถานะที่ถูกสร้างขึ้นจากการกำหนดสถานะที่กำลังพิจารณา 
ตัวอย่างเช่น ถ้ากำหนดให้สถานะที่กำลังพิจารณาเป็น $(1,1)$
ค่าของ $N$ จะมีค่าได้ดังนี้ $\{(3,1),(1,4),(0,1),(1,0),(2,0),(0,2)\}$
ซึ่งค่าของ $N$ จะได้มาจากกฎของการเปลี่ยนแปลงสถานะที่อธิบายตามสมการ \ref{eq_wj3}
จนถึง สมการ \ref{eq_wje4} โดยใช้ $(w_3,w_4)$ เป็นค่า $(1,1)$
}
%

\begin{algorithm}
\lmr
\caption{Water Jug Problem from $(0,0)$ to $(w_3,2)$ \label{al_wj} }
\begin{algorithmic}[1]
\Function{WaterJug BFS}{}
\State $\textit{W} \gets \{(0,0)\}$
\State $\textit{C} \gets \{\}$
\Repeat
  \State $(w_3,w_4) \gets pop(W)$
  \State $\textit{extend}(C,\{(w_3,w_4)\})$
  \State $N \gets \textit{generate}(w_3,w_4)$
  
  \State $\textit{extend}(W,\{k \in N | k \notin (W \cup C) \})$
\Until{$w_4 = 2 \textit{~or~} W = \emptyset$}
\If{$w_4 = 2$} 
\State \Return true;
\EndIf
\State \Return false;
\EndFunction
\end{algorithmic}
\end{algorithm}

