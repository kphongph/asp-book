\chapter{บทนำ}
\par{
Answer Set Programming (ASP) \index{Answer Set Programming}
เป็นภาษาโปรแกรมเชิงประกาศ 
(Declarative Programming Language)
\index{Declarative Programming Language}
มีลักษณะการเขียนแตกต่างจะภาษาโปรแกรมเชิงกระบวนการ 
(Imperative Programming) 
\index{Imperative Programming Language}
ซึ่งจะเน้นเรื่องลำดับขั้นตอนการทำงานของเครื่องจักร 
ตัวอย่างของภาษาเชิงกระบวนการคือ ภาษาจาวา ภาษาซี เป็นต้น
ในขณะที่ภาษาโปรแกรมเชิงกระบวนการเน้นขั้นตอนการทำงาน 
การภาษาโปรแกรมเชิงประกาศนั้นจะเน้นที่จะอธิบายว่าปัญหาคืออะไร 
แล้วให้เครื่องจักรทำงานแก้ปัญหาดังกล่าวเองโดยไม่จำเป็นต้องบอกให้เครื่องจักรทำงานอย่างไรบ้าง
เพื่อเข้าใจในความแตกต่างของการพัฒนาภาษาทั้งสองรูปแบบ 
เราจะศึกษาตัวอย่างปัญหาเหยือกน้ำ
จากนั้นจะเริ่มอธิบายวิธีการแก้ปัญหาโดยวิธีการโปรแกรมแบบเชิงกระบวนการ
และตามด้วยการโปรแกรมเชิงประกาศ
}
%
\section{ปัญหาเหยือกน้ำ}
\label{sec_wjp}
%
\par{
ปัญหาเหยือกน้ำ (Water Jug Problem) 
\index{Water Jug Problem} กำหนดให้มีเหยือกน้ำ 2 ใบ
มีขนาดความจุ 4 ลิตร กับ 3 ลิตร ตามลำดับ
โดยที่เหยือกน้ำทั้งสองใบไม่มีขีดสำหรับวัดความจุของน้ำ
เป็นไปได้หรือไม่ว่าจะให้มีน้ำเหลืออยู่จำนวน 2 ลิตรในเหยือกขนาด 4 ลิตร
ถ้ากำหนดให้มีน้ำสำหรับใส่ในแต่ละเหยือกได้ตลอด
}
%
\par{
เราจะเริ่มด้วยแนวทางในการแก้ปัญหาเหยือกน้ำโดยไม่เกี่ยวข้องกับการโปรแกรมก่อน
ในการแก้ปัญหาจากตัวอย่างปัญหาเหยือกน้ำ
อันดับแรกคือการอธิบายสถานะของปัญหา
ซึ่งสถานะจะต้องรวมตัวแปรทุกตัวที่มีผลต่อผลลัพธ์ของปัญหา
จากปัญหาดังกล่าวสถานะของปัญหาสามารถอธิบายได้ด้วยจำนวนน้ำที่อยู่ในเหยือกน้ำแต่ละใบ
ฉะนั้นกำหนดให้ $w_3$ และ $w_4$ 
เป็นปริมาณน้ำที่อยู่ในเหยือกขนาด 3 และ 4 ลิตรตามลำดับ
ฉะนั้นสถานะของปัญหา ณ เวลา $t$ ใด ๆ 
สามารถอธิยายได้ด้วยปริมาณของน้ำในเหยือกทั้งสองใบในลักษณะเป็นลำดับดังนี้ 
$(w_3, w_4)_t$
}
%
\par{
เมื่อได้สถานะในการอธิบายลำดับของปัญหาแล้ว 
เราจะเริ่มกำหนดหรืออธิบายการเปลี่ยนสถานะของปัญหาจากสถานะหนึ่งไปอีกสถานะหนึ่ง
จากปัญหาดังกล่าว 
ถ้ากำหนดให้สถานะปัจจุบันคือ $(w_3, w_4)_t$ 
การเปลี่ยนสถานะหรือการเปลี่ยนแปลงปริมาณน้ำในแต่ละใบจะเกิดขึ้นได้สามแบบ
คือ (1) การเติมน้ำให้เต็มสำหรับเหยือกที่ยังมีน้ำไม่เต็ม
(2) การเทน้ำจากเหยือกใบหนึ่งไปยังอีกใบหนึ่ง และ (3) การเทน้ำในเหยือกทิ้ง
กรณีที่เป็นไปได้ทั้งหมดของ $(w_3',w_4')_{t+1}$ 
สามารถแสดงได้ดังต่อไปนี้
\begin{enumerate}
\item การเพิ่มน้ำจากแหล่งน้ำให้กับเหยือกน้ำใบใดใบหนึ่งจนเต็ม 
\begin{eqnarray}
(w_3',w_4') \leftarrow (3, w_4) \label{eq_wj3} \\
(w_3',w_4') \leftarrow (w_3, 4) \label{eq_wj4}
\end{eqnarray}
\item การถ่ายเทน้ำจากเหยือก 4 ลิตรไปยังเหยือก 3 ลิตร
\begin{itemize}
\item ถ้า $w_3 + w_4 \le 3$ 
\begin{eqnarray}
(w_3',w_4') \leftarrow (w_3+w_4, 0) \label{eq_wj4to3_1}
\end{eqnarray}
\item ถ้า $w_3 + w_4 > 3$ 
\begin{eqnarray}
(w_3',w_4') \leftarrow (3, w_4-(3-w_3)) \label{eq_wj4to3_2}
\end{eqnarray}
\end{itemize}
\item การถ่ายเทน้ำจากเหยือก 3 ลิตรไปยังเหยือก 4 ลิตร
\begin{itemize}
\item ถ้า $w_3 + w_4 \le 4$ 
\begin{eqnarray}
(w_3',w_4') \leftarrow (0, w_3+w_4) \label{eq_wj3to4_1}
\end{eqnarray}
\item ถ้า $w_3 + w_4 > 4$ 
\begin{eqnarray}
(w_3',w_4') \leftarrow (w_3-(4-w_4),4) \label{eq_wj3to4_2}
\end{eqnarray}
\end{itemize}
\item การเทน้ำทั้งหมดในเหยือกทิ้ง
\begin{eqnarray}
(w_3',w_4') \leftarrow (0, w_4) \label{eq_wje3} \\
(w_3',w_4') \leftarrow (w_3, 0) \label{eq_wje4}
\end{eqnarray}
\end{enumerate}
%
ถ้ากำหนดให้ $(w_3,w_4)_t = (0,0)_0$ จากการเปลี่ยนสถานะข้างต้น $(w_3',w_4')_{t+1}$ 
จะสามารถเป็นหนึ่งในสมาชิกของเซตต่อไปนี้ $\{(0,0)_1,(3,0)_1,(0,4)_1\}$ ซึ่ง 
$(0,0)_1$ เกิดขึ้นได้จากหนึ่งในกฏที่อธิบายด้วยสมการต่อไปนี้ สมการที่ (\ref{eq_wj3to4_1})  
สมการที่ (\ref{eq_wj4to3_1}) สมการที่ (\ref{eq_wje3}) และ 
สมการที่ (\ref{eq_wje4})
ส่วน $(3,0)_1$ และ $(0,4)_1$ เกิดขึ้นจาก 
สมการที่ (\ref{eq_wj3}) และ สมการที่ (\ref{eq_wj4}) ตามลำดับ
}
%
\par{
จะเห็นได้ว่าเมื่อพิจารณาเฉพาะการเปลี่ยนสถานะของปริมาณน้ำ สถานะที่ $(0,0)_0$ กับ สถานะที่ $(0,0)_1$
ไม่มีข้อแตกต่างกันเนื่องจากทั้งสองมีปริมาณน้ำในแต่ละเหยือกเท่ากัน  
ฉะนั้นเราจะไม่คิดเรื่องของเวลาสำหรับระบุลำดับของการเปลี่ยนแปลงสถานะของปัญหา 
(สาเหตุที่ผู้แต่งใช้ $t$ ก่อนหน้านี้เพื่อที่จะอธิบายลำดับของการเปลี่ยนแปลงสถานะของปัญหา)
จากการกำหนดรูปแบบสถานะ 
เราสามารถสร้างการกราฟแสดงสถานะทั้งหมดที่เป็นไปได้
ดังแสดงในรูปภาพที่ \ref{fg_allstatewj}
และตัวอย่างของการเปลี่ยนแปลงสถานะจากเริ่มต้นไปยังสถานะที่ต้องการได้ดังแสดงในรูปภาพที่ \ref{fg_transwj_1}
}
%
%
\begin{figure}[h!]
\centering
\begin{pspicture}(0,-0.5)(6,5)
\rput(0,4.5){\rnode{c00}{$(0,0)$}}
\rput(1.5,4.5){\rnode{c01}{$(0,1)$}}
\rput(3,4.5){\rnode{c02}{$(0,2)$}}
\rput(4.5,4.5){\rnode{c03}{$(0,3)$}}
\rput(6,4.5){\rnode{c04}{$(0,4)$}}
%
\rput(0,3){\rnode{c10}{$(1,0)$}}
\rput(1.5,3){\rnode{c11}{$(1,1)$}}
\rput(3,3){\rnode{c12}{$(1,2)$}}
\rput(4.5,3){\rnode{c13}{$(1,3)$}}
\rput(6,3){\rnode{c14}{$(1,4)$}}
%
\rput(0,1.5){\rnode{c20}{$(2,0)$}}
\rput(1.5,1.5){\rnode{c21}{$(2,1)$}}
\rput(3,1.5){\rnode{c22}{$(2,2)$}}
\rput(4.5,1.5){\rnode{c23}{$(2,3)$}}
\rput(6,1.5){\rnode{c24}{$(2,4)$}}
%
\rput(0,0){\rnode{c30}{$(3,0)$}}
\rput(1.5,0){\rnode{c31}{$(3,1)$}}
\rput(3,0){\rnode{c32}{$(3,2)$}}
\rput(4.5,0){\rnode{c33}{$(3,3)$}}
\rput(6,0){\rnode{c34}{$(3,4)$}}
%
\end{pspicture}
\caption{
แสดงสถานะที่เป็นไปได้ทั้งหมดของปริมาณน้ำในเหยือกแต่ละใบ 
ในรูปแบบ $(w_3,w_4)$ โดยที่ $w_3$ และ $w_4$ 
แสดงปริมาณน้ำเป็นลิตรในเหยือกขนาดความจุ 3 ลิตร และ ขนาดความจุ 
4 ลิตรตามลำดับ
}
\label{fg_allstatewj}
\end{figure}
%
\par{
จากรูปภาพที่ \ref{fg_transwj_1} 
เนื่องจากเราสามารถหาการเปลี่ยนแปลงสถานะจาก $(0,0)$ ไปยัง $(3,2)$ 
ได้
ซึ่งเป็นตอบปัญหาว่าสามารถมีน้ำจำนวน 2 ลิตรเหลือให้เหยือกขนาด 4 ลิตรได้โดยเริ่มต้นที่ไม่มีน้ำอยู่ในเหยือกทั้งสองใบเลย
}
%
\begin{figure}[t]
\centering
\begin{pspicture}(0,-0.5)(6,5)
\rput(0,4.5){\rnode{c00}{$(0,0)$}}
\rput(1.5,4.5){\rnode{c01}{$(0,1)$}}
\rput(3,4.5){\rnode{c02}{$(0,2)$}}
\rput(4.5,4.5){\rnode{c03}{$(0,3)$}}
\rput(6,4.5){\rnode{c04}{$(0,4)$}}
%
\rput(0,3){\rnode{c10}{$(1,0)$}}
%\rput(1.5,3){\rnode{c11}{$(1,1)$}}
%\rput(3,3){\rnode{c12}{$(1,2)$}}
%\rput(4.5,3){\rnode{c13}{$(1,3)$}}
\rput(6,3){\rnode{c14}{$(1,4)$}}
%
\rput(0,1.5){\rnode{c20}{$(2,0)$}}
%\rput(1.5,1.5){\rnode{c21}{$(2,1)$}}
%\rput(3,1.5){\rnode{c22}{$(2,2)$}}
%\rput(4.5,1.5){\rnode{c23}{$(2,3)$}}
\rput(6,1.5){\rnode{c24}{$(2,4)$}}
%
\rput(0,0){\rnode{c30}{$(3,0)$}}
\rput(1.5,0){\rnode{c31}{$(3,1)$}}
\rput(3,0){\rnode{c32}{$(3,2)$}}
\rput(4.5,0){\rnode{c33}{$(3,3)$}}
\rput(6,0){\rnode{c34}{$(3,4)$}}
%
\psset{nodesep=3pt,arrowsize=3pt 3}
\ncarc[arcangleB=30,arcangleA=30,linestyle=solid]{->}{c00}{c04}\mput*{1}
\ncline[linestyle=solid]{->}{c04}{c31}\mput*{2}
\ncline[linestyle=solid]{->}{c31}{c01}\mput*{3}
\ncline[linestyle=solid]{->}{c01}{c10}\mput*{4}
\ncline[linestyle=solid]{->}{c10}{c14}\mput*{5}
\ncline[linestyle=solid]{->}{c14}{c32}\mput*{6}
\end{pspicture}
\caption{แสดงการเปลี่ยนสถานะของปริมาณน้ำโดยเริ่มต้นจากที่มีน้ำ 0 
ลิตรบรรจุอยู่ในเหยือกทั้งสองใบ ไปยังคำตอบที่ต้องการคือมีน้ำเหลืออยู่ 2
ลิตรในเหยือกขนาด 4 ลิตร
โดยที่เราจะพิจารณาเฉพาะสถานะที่ไม่ซ้ำกันเท่านั้น 
ตัวอย่างเช่น จาก $(0,4)$ สามารถเปลี่ยนเป็น $(0,0)$ หรือ $(3,4)$ ได้
แต่เนื่องจากทั้งสองสถานะมีอยู่แล้วจึงไม่มีความจำเป็นต้องสร้างสถานะดังกล่าวใหม่}
\label{fg_transwj_1}
\end{figure}
%
%
%
\section{การโปรแกรมเชิงกระบวนการ}
%
%
%
\par{
ในเนื้อหาส่วนนี้เราจะใช้แนวคิดการโปรแกรมเชิงกระบวนการเพื่อพัฒนาชุดคำสั่งที่ใช้ในการแก้ปัญหาจากตัวอย่างปัญหาเหยือกน้ำซึ่งอธิยายก่อนหน้านี้ในหัวข้อที่ \ref{sec_wjp}
โดยที่การที่จะตอบปัญหาเหยือกน้ำได้เราจำเป็นต้องหาเส้นทางการเปลี่ยนสถานะจากจุดต้้งต้นไปยังสถานะที่ต้องการ 
ถ้าเส้นทางดังกล่าวสามารถเกิดขึ้นจากการเปลี่ยนสถานะที่กำหนดไว้คำตอบของปัญหาคือเป็นไปได้
ในทางกลับกันเราจำเป็นต้องการันตีว่าไม่มีเส้นทางดังกล่าวอยู่จริงถึงจะตอบได้ว่าเป็นไปไม่ได้
จะแนวคิดการแก้ปัญหาข้างต้นสามารถเปลี่ยนให้เป็น 
การค้นหาเส้นทางจากจุดหนึ่งไปยังอีกจุดหนึ่งได้ 
ซึ่งกระบวนการดังกล่าวสามารถอธิบายได้เป็นลำดับขั้นต่อไปนี้ 
%
\begin{enumerate}
\item \label{enum_wj_1} การกำหนดค่าเริ่มต้นของสถานะ $(w_3,w_4)$ ในที่นี่คือ $(0,0)$ 
\item ตรวจสอบว่า $(w_3,w_4)$ เป็นสถานะที่ต้องการหรือไม่ ในที่นี่คือ $(w_3,2)$ 
ถ้าใช่หยุดการทำงานแล้วตอบว่าทำได้
\item ทำการสร้างสถานะที่เป็นไปได้ทั้งหมด $N$ โดยที่ $N = generate(w_3,w_4)$
\item จากสถานะใน $N$ เพิ่มสถานะที่ยังไม่ได้ถูกพิจารณาแล้วนำไปเก็บไว้ที่ $W$
\item เปลี่ยนสถานะตัวแรกของ $W$ ให้เป็นค่าเริ่มต้นตัวใหม่แล้วเริ่มทำ (\ref{enum_wj_1}) 
ถ้าไม่มีสมาชิกเหลือใน $W$ คำตอบคือไม่สามารถทำได้
\end{enumerate}
%
จากลำดับขั้นตอนดังกล่าวข้างต้นเราสามารถพัฒนาโปรแกรมเชิงกระบวนการได้
ดังแสดงใน 
การแสดงขั้นตอนวิธีการ (Algorithm) \ref{al_wj}
กำหนดให้ $W$ เป็นเซตของสถานะที่รอการพิจารณา $C$ เป็นเซตของสถานะที่พิจารณาเรียบร้อยแล้ว
โดยที่ $N$ เป็นเซตของสถานะที่ถูกสร้างขึ้นจากการกำหนดสถานะที่กำลังพิจารณา 
ตัวอย่างเช่น ถ้ากำหนดให้สถานะที่กำลังพิจารณาเป็น $(1,1)$
ค่าของ $N$ จะมีค่าได้ดังนี้ $\{(3,1),(1,4),(0,1),(1,0),(2,0),(0,2)\}$
ซึ่งค่าของ $N$ จะได้มาจากกฎของการเปลี่ยนแปลงสถานะที่อธิบายตามสมการ \ref{eq_wj3}
จนถึง สมการ \ref{eq_wje4} โดยใช้ $(w_3,w_4)$ เป็นค่า $(1,1)$
}
%

%
%
\begin{algorithm}[t]
\lmr
\caption{Water Jug Problem from $(0,0)$ to $(w_3,2)$ \label{al_wj} }
\begin{algorithmic}[1]
\Function{WaterJug BFS}{}
\State $\textit{W} \gets \{(0,0)\}$
\State $\textit{C} \gets \{\}$
\Repeat
  \State $(w_3,w_4) \gets pop(W)$
  \If{$w_4 \neq 2$} 
    \State $\textit{append}((w_3,w_4),C)$
    \State $N \gets \textit{generate}(w_3,w_4)$
    \For{$(w_3',w_4') \in N$}
      \If{$(w_3',w_4') \notin (W \cup C)$}
        \State $\textit{append}((w_3',w_4'),W)$
      \EndIf
    \EndFor
  \EndIf
\Until{$w_4 = 2 \textit{~or~} W = \emptyset$}
\If{$w_4 = 2$} 
\State \Return true;
\EndIf
\State \Return false;
\EndFunction
\end{algorithmic}
\end{algorithm}
%
%
\par{
จะเห็นได้ว่าการพัฒนาโปรแกรมเชิงกระบวนการจะเน้นเรื่องลำดับขั้นตอนการทำงานอย่างชัดเจน
ซึ่งการกำหนดลำดับขั้นตอนก่อนหลังมีส่วนสำคัญในการหาคำตอบ 
ถ้ามีการเปลี่ยนลำดับของการพิจารณาการเปลี่ยนสถานะ 
(ลำดับของสมการ \ref{eq_wj3} กับ \ref{eq_wj4}) 
อาจนำมาซึ่งลำดับของการพิจารณาที่แตกต่างกันได้ ดังแสดงในรูปภาพที่ \ref{fg_transwj_2}
เป็นต้น
}
%
\begin{figure}[t]
\centering
\begin{pspicture}(0,-0.5)(6,5)
\rput(0,4.5){\rnode{c00}{$(0,0)$}}
\rput(1.5,4.5){\rnode{c01}{$(0,1)$}}
\rput(3,4.5){\rnode{c02}{$(0,2)$}}
\rput(4.5,4.5){\rnode{c03}{$(0,3)$}}
\rput(6,4.5){\rnode{c04}{$(0,4)$}}
%
\rput(0,3){\rnode{c10}{$(1,0)$}}
%\rput(1.5,3){\rnode{c11}{$(1,1)$}}
%\rput(3,3){\rnode{c12}{$(1,2)$}}
%\rput(4.5,3){\rnode{c13}{$(1,3)$}}
\rput(6,3){\rnode{c14}{$(1,4)$}}
%
\rput(0,1.5){\rnode{c20}{$(2,0)$}}
%\rput(1.5,1.5){\rnode{c21}{$(2,1)$}}
%\rput(3,1.5){\rnode{c22}{$(2,2)$}}
%\rput(4.5,1.5){\rnode{c23}{$(2,3)$}}
\rput(6,1.5){\rnode{c24}{$(2,4)$}}
%
\rput(0,0){\rnode{c30}{$(3,0)$}}
\rput(1.5,0){\rnode{c31}{$(3,1)$}}
\rput(3,0){\rnode{c32}{$(3,2)$}}
\rput(4.5,0){\rnode{c33}{$(3,3)$}}
\rput(6,0){\rnode{c34}{$(3,4)$}}
%
\psset{nodesep=3pt,arrowsize=3pt 3}
\ncarc[arcangleB=-35,arcangleA=-35,linestyle=solid]{->}{c00}{c30}\mput*{1}
\ncline[linestyle=solid]{->}{c30}{c03}\mput*{2}
\ncline[linestyle=solid]{->}{c03}{c33}\mput*{3}
\ncline[linestyle=solid]{->}{c33}{c24}\mput*{4}
\ncline[linestyle=solid]{->}{c24}{c20}\mput*{5}
\ncline[linestyle=solid]{->}{c20}{c02}\mput*{6}
\end{pspicture}
\caption{แสดงการเปลี่ยนสถานะของปริมาณน้ำโดยเริ่มต้นจากที่มีน้ำ 0 
ลิตรบรรจุอยู่ในเหยือกทั้งสองใบ 
โดยมีลำดับเส้นทางแตกต่างจาก รูปภาพที่ \ref{fg_transwj_1}}
\label{fg_transwj_2}
\end{figure}
%
\par{
การแสดงขั้นตอนวิธีการ (Algorithm) \ref{al_wj}
เราสามารถสร้างโปรแกรมการทำงานแบบเชิงกระบวนการด้วยภาษา Javascript บน NodeJS 
ที่ทำงานได้บนเครื่องคอมพิวเตอร์ทั่วไปที่มี NodeJS runtime อยู่ได้ดังแสดงต่อไปนี้
%
\begin{lstlisting}[
  xleftmargin=2em,framexleftmargin=1.5em,
  language=JavaScript,
  backgroundcolor=\color{lightgray},
  extendedchars=true,
  basicstyle=\footnotesize\ttfamily,
  showstringspaces=false,
  showspaces=false,
  numbers=left,
  numberstyle=\footnotesize\ttfamily,
  numbersep=5pt,
  tabsize=2,
  breaklines=true,
  showtabs=false,
  captionpos=b
]
var generate = function(ws) {
  var N = [];
  N.push({w3:3,w4:ws.w4});
  N.push({w3:ws.w3,w4:4});
  N.push({w3:0,w4:ws.w4});
  N.push({w3:ws.w3,w4:0});
  N.push({w3:ws.w3+ws.w4,w4:0});
  N.push({w3:0,w4:ws.w3+ws.w4});
  N.push({w3:3,w4:ws.w4-(3-ws.w3)});
  N.push({w3:ws.w3-(4-ws.w4),w4:4});
  return N;
}

var main = function(ws) {
  var C = [];
  var W = [ws];
  while(W.length!=0) {
    cws = W.pop();
    if(cws.w4==2) break;
    C.push(cws);
    N = generate(cws);
    for(var i=0;i<N.length;i++) {
      var e = N[i];
      if(e.w3>=0 && e.w3<=3 && e.w4>=0 && e.w4<=4) {
        var explored = false;
        for(var j=0;j<C.length;j++) {
          var k=C[j];
          if(k.w3==e.w3 && k.w4==e.w4) {
            explored = true;
            break;
          }
        }
        if(!explored) {
          W.push(e);
        }
      }
    }
  }
  if(cws.w4==2) {
    console.log('Yes');
  } else {
    console.log('No');
  }
}
\end{lstlisting}
%
การทำงานของโปรแกรมจะประกอบไปด้วย 2 ส่วนหลักคือ
การควบคุมการทำงานของการค้นหาสถานะที่ต้องการซึ่งอยู่ในส่วนของ \texttt{main}
กับ ส่วนการสร้างสถานะที่สามารถเป็นไปได้ทั้งหมดจากสถานะที่กำหนดให้ \texttt{generate}
ถึงแม้ว่าโปรแกรมดังกล่าวจะมีจำนวนบรรทัดที่สั้นสามารถอ่าน
และทำความเข้าใจได้โดยง่ายในเวลาไม่นานนักสำหรับนักโปรแกรมที่มีความชำนาญ
แต่เมื่อพิจารณาการเขียนโปรแกรมที่มีขนาดใหญ่หรือความซับซ้อนมากกว่านี้อาจจะต้องใช้เวลานาน
ในการทำความเข้าใจและสามารถพิจารณาความถูกต้องของการพัฒนาโปรแกรมได้ 
ในเรื่องของการพิจารณาความถูกต้องในที่นี่ 
หมายถึงการเขียนโปรแกรมให้เป็นไปตามขั้นตอนกระบวนการที่ระบุไว้
ผู้อ่านตัวโปรแกรมต้องพิจารณาในแต่ละขั้นตอนของโปรแกรมว่ามีการเปลี่ยนสถานะยังไง
มีผลยังไงกับผลลัพธ์ของการทำงาน
มีความสอดคล้องหรือขัดแย้งกับ การแสดงขั้นตอนวิธีการ (ดังแสดงใน Algorithm \ref{al_wj})
หรือไม่
ซึ่งเป็นเรื่องที่ทำได้แต่ไม่ง่าย เนื่องจากการแสดงขั้นตอนวิธีการไม่เหมือนกันกับการเขียนโปรแกรมจริง ๆ
ดังตัวอย่างที่ยกมาก่อนหน้านี้เป็นต้น
}
%
%
%
%
\section{การโปรแกรมเชิงประกาศ}
%
%
%


\begin{lstlisting}[xleftmargin=1em]
r(-10..10).
l(0..4).
w(0,0).
{w(0,Y),w(X,0),w(3,Y),w(X,4)} :- w(X,Y),l(X;Y).
{w(X+Y,0),w(0,X+Y)} :- w(X,Y),l(X;Y).
{w(X-(4-Y),4),w(3,Y-(3-X))} :- w(X,Y),l(X;Y).
goal :- w(X,2),l(X).
:-w(X,Y),r(X;Y),X>3.
:-w(X,Y),r(X;Y),X<0.
:-w(X,Y),r(X;Y),Y>4.
:-w(X,Y),r(X;Y),Y<0.
:-not goal.
\end{lstlisting}





