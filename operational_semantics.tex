\chapter{ความหมายเชิงกระบวนการ}
%
\par{
เนื้อหาในส่วนนี้จะกล่าวถึงการให้ความหมายโดยวิธีการเชิงกระบวนการ 
(Operational Semantics)
ซึ่งเป็นการกำหนดความหมายให้ภาษาคอมพิวเตอร์แบบหนึ่งที่เหมาะสมกับการโปรแกรมเชิงกระบวน
}
%
\par{
หลักการพื้นฐานของการให้ความหมายเชิงกระบวนการ 
คือการใช้เปลี่ยนแปลงของสถานะการทำงานของโปรแกรมเพื่อกำหนดความหมายให้กับโปรแกรม
ฉะนั้นสถานะของโปรแกรม (State) 
จะต้องถูกอธิบายมาอย่างชัดเจนว่าประกอบด้วยอะไรบ้าง 
รวมทั้ง การเปลี่ยนแปลงของสถานะของโปรแกรม (State transition) 
เกิดขึ้นได้อย่างใดบ้าง
}
%
\par{
\begin{example}
%
พิจารณาตัวอย่างของส่วนของโปรแกรมที่อธิบายด้วย BNF 
ต่อไปนี้
\begin{grammar}
<e> ::= `true' | `false' | `not' <e> | `if' <e> <e> <e>
\end{grammar}
\end{example}
}
%
กำหนดให้สถานะของโปรแกรมคือ $Exp$ โดยที่ 
$\langle e \rangle \in Exp$ 
%
\par{
การเปลี่ยนสถานะของโปรแกรม ($\mapsto$)
เป็นความสัมพันธ์ระหว่างสถานะของโปรแกรมก่อนการเปลี่ยนแปลง ($e$) 
กับสถานะของโปรแกรมหลังการเปลี่ยนแปลง ($e'$)
ฉะนั้น การเปลี่ยนสถานะของโปรแกรมสามารถเขียนในรูปคณิตศาสตร์ได้ 
$$
e \mapsto e' \subseteq Exp~\times~Exp
$$
จากโปรแกรมข้างต้นกำหนดให้มีการเปลี่ยนสถานะโปรแกรมทั้งหมด 4 รูปแบบต่อไปนี้
%
\begin{enumerate}
\item \texttt{not} \texttt{true} $\mapsto$ \texttt{false}
\item \texttt{not} \texttt{false} $\mapsto$ \texttt{true}
\item \texttt{if} \texttt{true} $e_1$ $e_2$ $\mapsto$ $e_1$
\item \texttt{if} \texttt{false} $e_1$ $e_2$ $\mapsto$ $e_2$
\end{enumerate}
%
จะเห็นได้ว่านอกจากจะบอกลักษณะของสถานะที่เปลี่ยนแปลงไปของการเปลี่ยนแปลงสถานะแล้ว 
ความหมายยังถูกแสดงออกมาในแต่ละการเปลี่ยนแปลงอีกด้วย ตัวอย่างเช่น
\texttt{if} \texttt{true} $e_1$ $e_2$ $\mapsto$ $e_1$
สามารถอธิบายอยู่ในรูปของความหมายของโปรแกรมได้ว่า 
ถ้าตัวตรวจสอบของการตัดสินใจ 
(\texttt{if true then} $e_1$ \texttt{else} $e_2$)
แบบมีค่าความจริงเป็นจริงแล้ว 
ความหมายของประโยคดังกล่าวจะอยู่เฉพาะในส่วนแรก ($e_1$) เท่านั้น
}
%
\par{
เมื่อมีการกำหนดประโยคสัญลักษณ์ด้วยกระบวนเปลี่ยนแปลงสถานะของโปรแกรมที่ถูกกำหนดมาดังกล่าวข้างต้น
จะทำให้ทราบถึงความหมายของประโยคสัญลักษณ์ดังกล่าวได้ดังเช่นในตัวอย่างต่อไปนี้
\begin{align*}
&~~\texttt{if (not true) (not false) (if true (not true) false)} \\
\mapsto &~~\texttt{if false (not false) (if true (not true) false)} \\
\mapsto &~~\texttt{if true (not true) false} \\
\mapsto &~~\texttt{not true} \\
\mapsto &~~\texttt{false} 
\end{align*}
}
